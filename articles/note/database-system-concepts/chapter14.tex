\chapter{事务(Transaction)}
%---------- 14.1 ----------------
\section{事务概念}
\begin{flushleft}
\textbf{事务}是访问并可能更新各种数据项的一个程序执行\textbf{单元}(unit)。有以下几个特性(ACID):\\
\end{flushleft}

\begin{itemize}
  \item 1.原子性(Atomicity):事务的所有操作在数据库中要么全部正确反映出来,要么完全不反映。
  \item 2.一致性(Consistency):隔离执行事务时(换言之,在没有其他事务并发执行的情况下)保持数据库的一致性。即如果一个事务作为原子从一个一致的数据库状态开始独立地运行,则事务结束时数据库也必须再次是一致的。
  \item 3.隔离性(Isolation):尽管多个事务可能并发执行,但系统保证,对于任何一对事务$T_{i}$和$T_{j}$,在$T_{i}$看来,$T_{j}$或者$T_{i}$开始之前已经完成执行,或者在$T_{i}$完成之后开始执行。因此,每个事务都感觉不到系统中有其他事务在并发地执行。
  \item 4.持久性(Durability):一个事务成功完成后,它对数据库的改变必须是永久的,即使出现系统故障。
\end{itemize}


