\chapter{在Direct3D中绘制2(Drawing in Direct3D Part 2)}
\begin{flushleft}
本章介绍我们将在本书其余部分使用的一些绘图模式。 本章首先介绍一种绘图优化的方法,我们将其称为“帧资源”。使用帧资源,我们修改渲染循环,这样我们就不必每帧刷新命令队列; 这提高了CPU和GPU的利用率。 接下来,我们介绍渲染项的概念,并解释我们如何根据更新频率划分常量数据。 此外,我们更详细地检查根签名,并了解其他根参数类型:根描述符和根常量。 最后,我们展示了如何绘制一些更复杂的对象; 到本章结束时,您将能够绘制类似丘陵和山谷,圆柱体,球体和动画波浪模拟的表面。\\
~\\
{\large Objectives:}
\begin{itemize}
    \item 了解对渲染过程的修改,不要求我们每帧刷新命令队列,从而提高性能。
    \item 了解其他两种类型的根签名参数类型:根描述符和根常量。
    \item 探索如何在程序上生成和绘制常见的几何形状,如网格,圆柱体和球体。
    \item 了解我们如何在CPU上设置顶点动画并使用动态顶点缓冲区将新顶点位置上传到GPU。
\end{itemize}
\end{flushleft}

%---- 7.1 ------
\section{帧资源(Frame Resources)}
\begin{flushleft}
回忆 4.2 节,CPU和GPU并行工作。 CPU构建并提交命令列表(除了其他CPU工作之外),GPU处理命令队列中的命令。 目标是让CPU和GPU忙碌,以充分利用系统上可用的硬件资源。 到目前为止,在我们的演示中,我们已经每帧同步CPU和GPU一次。 两个例子解释这种同步的必要性:\\
\begin{itemize}
    \item 1.在GPU完成执行命令之前,不能重置命令分配器(command allocator)。 假设我们没有进行同步,以便CPU在GPU处理完当前帧n之前可以继续下一帧n + 1:如果CPU在帧n + 1中重置命令分配器,但GPU仍在处理命令 从第n帧开始,我们将清除GPU仍在使用的命令。
    \item 2.在GPU完成执行引用常量缓冲区的绘图命令之前,CPU无法更新常量缓冲区。 此示例对应于4.2.2节和图4.7中描述的情况。 假设我们没有进行同步,以便CPU在GPU完成处理当前帧n之前可以继续下一帧n + 1:如果CPU在帧n + 1中覆盖常量缓冲区数据,但GPU还没有 执行引用帧n中的常量缓冲区的绘制调用,然后常量缓冲区包含当GPU执行帧n的绘制调用时的错误数据。
\end{itemize}
因此,我们一直在每帧结束时调用 D3DApp::FlushCommandQueue,以确保GPU已完成执行帧的所有命令。 此解决方案有效,但由于以下原因效率低下:\\
\begin{itemize}
    \item 1.在帧开始时,GPU将不会有任何要处理的命令,因为我们等待清空命令队列。 它必须等到CPU构建并提交一些命令才能执行。
    \item 2.在帧结束时,CPU正在等待GPU完成处理命令。
\end{itemize}
所以每一帧,CPU和GPU都会在某些时候空闲。\\
~\\
该问题的一个解决方案是创建CPU修改每个帧所需的资源的循环数组。 我们称这种资源为帧资源,我们通常使用三个帧资源元素的循环数组。 对于帧n,CPU将循环通过帧资源队列以获得下一个可用(即,未被GPU使用)帧资源。 然后,CPU将执行任何资源更新,并在GPU处理先前帧时构建和提交帧n的命令列表。 然后CPU继续进行第n + 1帧并重复。 如果帧资源阵列有三个元素,这可以使CPU在GPU之前达到两帧,从而确保GPU保持忙碌状态。 下面是我们在本章中用于“Shapes”演示的帧资源类的示例。 由于CPU只需要在此演示中修改常量缓冲区,因此帧资源类仅包含常量缓冲区。\\
\end{flushleft}
\begin{lstlisting}
// Stores the resources needed for the CPU to build
// the command lists
// for a frame. The contents here will vary from app
// to app based on
// the needed resources.
struct FrameResource
{
public:
    FrameResource(ID3D12Device* device, UINT passCount, UINT objectCount);
    FrameResource(const FrameResource& rhs) = delete;
    FrameResource& operator=(const FrameResource& rhs) = delete;
    ~FrameResource();
    // We cannot reset the allocator until the GPU is
    // done processing the
    // commands. So each frame needs their own
    // allocator.
    Microsoft::WRL::ComPtr<ID3D12CommandAllocator> CmdListAlloc;
    
    // We cannot update a cbuffer until the GPU is done
    // processing the
    // commands that reference it. So each frame needs
    // their own cbuffers.
    std::unique_ptr<UploadBuffer<PassConstants>> PassCB = nullptr;
    std::unique_ptr<UploadBuffer<ObjectConstants>>
    ObjectCB = nullptr;
    // Fence value to mark commands up to this fence
    // point. This lets us
    // check if these frame resources are still in use
    // by the GPU.
    UINT64 Fence = 0;
};

FrameResource::FrameResource(ID3D12Device* device,
                             UINT passCount, UINT
                             objectCount)
{
    ThrowIfFailed(device->CreateCommandAllocator(
        D3D12_COMMAND_LIST_TYPE_DIRECT,
        IID_PPV_ARGS(CmdListAlloc.GetAddressOf())));
    PassCB = std::make_unique<UploadBuffer<PassConstants>>(
        device,
        passCount, 
        true);
    ObjectCB = std::make_unique<UploadBuffer<ObjectConstants>>(
        device,
        objectCount, true);
}
FrameResource::˜FrameResource() {}
\end{lstlisting}
\begin{flushleft}
然后,我们的应用程序类将三个帧资源的向量实例化,并设置成员变量以跟踪当前帧资源:\\
\end{flushleft}
\begin{lstlisting}
static const int NumFrameResources = 3;
std::vector<std::unique_ptr<FrameResource>> mFrameResources;
FrameResource* mCurrFrameResource = nullptr;
int mCurrFrameResourceIndex = 0;
void ShapesApp::BuildFrameResources()
{
    for(int i = 0; i < gNumFrameResources; ++i)
    {
        mFrameResources.push_back(std::make_unique<FrameResource>(
                                      md3dDevice.Get(), 
                                      1, 
                                      (UINT)mAllRitems.size()));
    }
}
\end{lstlisting}
\begin{flushleft}
现在,对于CPU帧n,算法的工作原理如下:\\
\end{flushleft}
\begin{lstlisting}
void ShapesApp::Update(const GameTimer& gt)
{
    // Cycle through the circular frame resource array.
    mCurrFrameResourceIndex = (mCurrFrameResourceIndex + 1) % NumFrameResources;
    mCurrFrameResource = mFrameResources[mCurrFrameResourceIndex];
    // Has the GPU finished processing the commands of
    // the current frame
    // resource. If not, wait until the GPU has
    // completed commands up to
    // this fence point.
    if(mCurrFrameResource->Fence != 0 &&
       mCommandQueue->GetLastCompletedFence() < mCurrFrameResource->Fence)
    {
        HANDLE eventHandle = CreateEventEx(nullptr, false, 
                                  false, EVENT_ALL_ACCESS);
        ThrowIfFailed(mCommandQueue->SetEventOnFenceCompletion(
             mCurrFrameResource->Fence, eventHandle));
        WaitForSingleObject(eventHandle, INFINITE);
        CloseHandle(eventHandle);
    }
    // […] Update resources in mCurrFrameResource (like cbuffers).
}
void ShapesApp::Draw(const GameTimer& gt)
{
    // […] Build and submit command lists for this frame.
    // Advance the fence value to mark commands up to
    // this fence point.
    mCurrFrameResource->Fence = ++mCurrentFence;
    // Add an instruction to the command queue to set a
    // new fence point.
    // Because we are on the GPU timeline, the new fence
    // point won’t be
    // set until the GPU finishes processing all the
    // commands prior to
    // this Signal().
    mCommandQueue->Signal(mFence.Get(), mCurrentFence);
    // Note that GPU could still be working on commands
    // from previous
    // frames, but that is okay, because we are not
    // touching any frame
    // resources associated with those frames.
}
\end{lstlisting}
\begin{flushleft}
请注意,此解决方案不会阻止等待。 如果一个处理器处理帧的速度比另一个处理器快得多,那么一个处理器最终将不得不等待另一个处理器赶上,因为我们不能让一个处理器远远超过另一个处理器。 如果GPU处理命令的速度比CPU提交工作的速度快,那么GPU将处于空闲状态。 一般来说,如果我们试图推动图形限制,我们希望避免这种情况,因为我们没有充分利用GPU。 另一方面,如果CPU总是以比GPU更快的速度处理帧,那么CPU将不得不在某个时刻等待。 这是理想的情况,因为GPU正在被充分利用; 额外的CPU周期总是可以用于游戏的其他部分,如AI,物理和游戏逻辑。\\
因此,如果多个帧资源不能阻止任何等待,它对我们有何帮助? 它可以帮助我们保持GPU的供给。 当GPU正在处理来自帧n的命令时,它允许CPU继续构建和提交帧n + 1和n + 2的命令。 这有助于保持命令队列非空,以便GPU始终有工作要做。
\end{flushleft}

%---- 7.2 ------
\section{渲染项(Render Items)}
\begin{flushleft}
绘制对象需要设置多个参数,例如绑定顶点和索引缓冲区,绑定对象常量,设置基本类型(primitive type)以及指定DrawIndexedInstanced参数。 当我们开始在场景中绘制更多对象时,创建一个存储绘制对象所需数据的轻量级结构会很有帮助。 这些数据因应用程序而异,因为我们添加了需要不同绘图数据的新功能。 我们将提交完整绘制所需的数据集称为渲染管道渲染项。 对于此演示,我们的RenderItem结构如下所示:\\
\end{flushleft}
\begin{lstlisting}
// Lightweight structure stores parameters to draw a shape.  This will
// vary from app-to-app.
struct RenderItem
{
    RenderItem() = default;

    // World matrix of the shape that describes the object's local space
    // relative to the world space, which defines the position, orientation,
    // and scale of the object in the world.
    XMFLOAT4X4 World = MathHelper::Identity4x4();

    // Dirty flag indicating the object data has changed 
    // and we need to update the constant buffer.
    // Because we have an object cbuffer for each FrameResource, 
    // we have to apply the update to each FrameResource.  
    // Thus, when we modify obect data we should set 
    // NumFramesDirty = gNumFrameResources so that each 
    // frame resource gets the update.
    int NumFramesDirty = gNumFrameResources;

    // Index into GPU constant buffer corresponding to 
    // the ObjectCB for this render item.
    UINT ObjCBIndex = -1;

    MeshGeometry* Geo = nullptr;

    // Primitive topology.
    D3D12_PRIMITIVE_TOPOLOGY PrimitiveType = D3D_PRIMITIVE_TOPOLOGY_TRIANGLELIST;

    // DrawIndexedInstanced parameters.
    UINT IndexCount = 0;
    UINT StartIndexLocation = 0;
    int BaseVertexLocation = 0;
};
\end{lstlisting}
\begin{flushleft}
我们的应用程序将根据需要绘制的方式维护渲染项目列表; 也就是说,需要不同PSO的渲染项目将保存在不同的列表中。\\
\end{flushleft}
\begin{lstlisting}
// List of all the render items.
std::vector<std::unique_ptr<RenderItem>> mAllRitems;
// Render items divided by PSO.
std::vector<RenderItem*> mOpaqueRitems;
std::vector<RenderItem*> mTransparentRitems;
\end{lstlisting}

%---- 7.3 ------
\section{传递常量(Pass Constants)}
\begin{flushleft}
NOTICE: Pass Constants 这是一整个名词。该常量用于存储额外信息\\
~\\
从上一节中可以看出,我们在FrameResource类中引入了一个新的常量缓冲区:
\end{flushleft}
\begin{lstlisting}
std::unique_ptr<UploadBuffer<PassConstants>> PassCB = nullptr;
\end{lstlisting}
\begin{flushleft}
在演示中,此缓冲区存储在给定渲染过程中固定的常量数据,例如眼睛位置,视图和投影矩阵,以及有关屏幕(渲染目标)尺寸的信息; 它还包括游戏计时信息,这是在着色器程序中可以访问的有用数据。 请注意,我们的演示不一定会使用所有这些常量数据,我们可以方便地使用这些数据,并且提供额外数据的成本很低。 例如,虽然我们现在不需要渲染目标大小,但是当我们实现一些后期处理效果时,将需要具有该信息。
\end{flushleft}
\begin{lstlisting}
cbuffer cbPass : register(b1)
{
    float4x4 gView;
    float4x4 gInvView;
    float4x4 gProj;
    float4x4 gInvProj;
    float4x4 gViewProj;
    float4x4 gInvViewProj;
    float3   gEyePosW;
    float    cbPerObjectPad1;
    float2   gRenderTargetSize;
    float2   gInvRenderTargetSize;
    float    gNearZ;
    float    gFarZ;
    float    gTotalTime;
    float    gDeltaTime;
};
\end{lstlisting}
\begin{flushleft}
我们还修改了每个对象常量缓冲区,仅存储与对象关联的常量。 到目前为止,我们与绘图对象关联的唯一常量数据是其世界矩阵:
\end{flushleft}
\begin{lstlisting}
cbuffer cbPerObject : register(b0)
{
    float4x4 gWorld;
};
\end{lstlisting}
\begin{flushleft}
这些更改是根据更新频率对常量进行分组。 每次通过常量只需要在每次渲染过程中更新一次,并且对象常量只需要在对象的世界矩阵发生变化时进行更改。如果我们在场景中有一个静态对象,就像一棵树,我们只需要将其世界矩阵设置一次到一个常量缓冲区,然后再也不要更新常量缓冲区。 在我们的演示中,我们实现了以下方法来处理每次传递和每个对象常量缓冲区的更新。Update方法中每帧调用一次这些方法。
\end{flushleft}
\begin{lstlisting}
void ShapesApp::UpdateObjectCBs(const GameTimer& gt)
{
    auto currObjectCB = mCurrFrameResource->ObjectCB.get();
    for(auto& e : mAllRitems)
    {
        // Only update the cbuffer data if the constants have changed.  
        // This needs to be tracked per frame resource.
        if(e->NumFramesDirty > 0)
        {
            XMMATRIX world = XMLoadFloat4x4(&e->World);

            ObjectConstants objConstants;
            XMStoreFloat4x4(&objConstants.World, XMMatrixTranspose(world));

            currObjectCB->CopyData(e->ObjCBIndex, objConstants);

            // Next FrameResource need to be updated too.
            e->NumFramesDirty--;
        }
    }
}

void ShapesApp::UpdateMainPassCB(const GameTimer& gt)
{
    XMMATRIX view = XMLoadFloat4x4(&mView);
    XMMATRIX proj = XMLoadFloat4x4(&mProj);

    XMMATRIX viewProj = XMMatrixMultiply(view, proj);
    XMMATRIX invView = XMMatrixInverse(&XMMatrixDeterminant(view), view);
    XMMATRIX invProj = XMMatrixInverse(&XMMatrixDeterminant(proj), proj);
    XMMATRIX invViewProj = XMMatrixInverse(&XMMatrixDeterminant(viewProj), viewProj);

    XMStoreFloat4x4(&mMainPassCB.View, XMMatrixTranspose(view));
    XMStoreFloat4x4(&mMainPassCB.InvView, XMMatrixTranspose(invView));
    XMStoreFloat4x4(&mMainPassCB.Proj, XMMatrixTranspose(proj));
    XMStoreFloat4x4(&mMainPassCB.InvProj, XMMatrixTranspose(invProj));
    XMStoreFloat4x4(&mMainPassCB.ViewProj, XMMatrixTranspose(viewProj));
    XMStoreFloat4x4(&mMainPassCB.InvViewProj, XMMatrixTranspose(invViewProj));
    mMainPassCB.EyePosW = mEyePos;
    mMainPassCB.RenderTargetSize = XMFLOAT2((float)mClientWidth, (float)mClientHeight);
    mMainPassCB.InvRenderTargetSize = XMFLOAT2(1.0f / mClientWidth, 1.0f / mClientHeight);
    mMainPassCB.NearZ = 1.0f;
    mMainPassCB.FarZ = 1000.0f;
    mMainPassCB.TotalTime = gt.TotalTime();
    mMainPassCB.DeltaTime = gt.DeltaTime();

    auto currPassCB = mCurrFrameResource->PassCB.get();
    currPassCB->CopyData(0, mMainPassCB);
}
\end{lstlisting}
\begin{flushleft}
我们相应地更新顶点着色器以支持这些常量缓冲区更改:\\
\end{flushleft}
\begin{lstlisting}
VertexOut VS(VertexIn vin)
{
    VertexOut vout;
    // Transform to homogeneous clip space.
    float4 posW = mul(float4(vin.PosL, 1.0f), gWorld);
    vout.PosH = mul(posW, gViewProj);
    // Just pass vertex color into the pixel shader.
    vout.Color = vin.Color;
    return vout;
}
\end{lstlisting}
\begin{flushleft}
这种调整每个顶点的额外矢量矩阵乘法在具有充足计算能力的现代GPU上可以忽略不计。\\
~\\
我们的着色器所期望的资源已经改变了; 因此,我们需要相应地更新根签名以获取两个描述符表(我们需要两个表,因为CBV将被设置为不同的频率 - 每次传递CBV仅需要在每个渲染过程中设置一次,而每个对象CBV需要 每个渲染项设置):
\end{flushleft}
\begin{lstlisting}
void ShapesApp::BuildRootSignature()
{
    CD3DX12_DESCRIPTOR_RANGE cbvTable0;
    cbvTable0.Init(D3D12_DESCRIPTOR_RANGE_TYPE_CBV, 1, 0);

    CD3DX12_DESCRIPTOR_RANGE cbvTable1;
    cbvTable1.Init(D3D12_DESCRIPTOR_RANGE_TYPE_CBV, 1, 1);

    // Root parameter can be a table, root descriptor or root constants.
    CD3DX12_ROOT_PARAMETER slotRootParameter[2];

    // Create root CBVs.
    slotRootParameter[0].InitAsDescriptorTable(1, &cbvTable0);
    slotRootParameter[1].InitAsDescriptorTable(1, &cbvTable1);

    // A root signature is an array of root parameters.
    CD3DX12_ROOT_SIGNATURE_DESC rootSigDesc(2, slotRootParameter, 0, nullptr, 
        D3D12_ROOT_SIGNATURE_FLAG_ALLOW_INPUT_ASSEMBLER_INPUT_LAYOUT);

    // create a root signature with a single slot which points to a descriptor range consisting of a single constant buffer
    ComPtr<ID3DBlob> serializedRootSig = nullptr;
    ComPtr<ID3DBlob> errorBlob = nullptr;
    HRESULT hr = D3D12SerializeRootSignature(&rootSigDesc, D3D_ROOT_SIGNATURE_VERSION_1,
        serializedRootSig.GetAddressOf(), errorBlob.GetAddressOf());

    if(errorBlob != nullptr)
    {
        ::OutputDebugStringA((char*)errorBlob->GetBufferPointer());
    }
    ThrowIfFailed(hr);

    ThrowIfFailed(md3dDevice->CreateRootSignature(
        0,
        serializedRootSig->GetBufferPointer(),
        serializedRootSig->GetBufferSize(),
        IID_PPV_ARGS(mRootSignature.GetAddressOf())));
}
\end{lstlisting}
\begin{flushleft}
NOTICE: 不要过多地使用着色器中的常量缓冲区数量。 [Thibieroz13]建议您将它们保持在5以下以保持性能。
\end{flushleft}

%---- 7.4 ------
\section{图形集合(Shape Geometry)}
\begin{flushleft}
在本节中,我们将展示如何为椭圆体,球体,圆柱体和圆锥体创建几何体。 这些形状对于绘制天空圆顶,调试,可视化碰撞检测和延迟渲染非常有用。 例如,您可能希望将所有游戏角色渲染为调试测试的球体。\\
~\\
我们将过程几何生成代码放在GeometryGenerator类(GeometryGenerator.h / .cpp)中。 GeometryGenerator是一个实用程序类,用于生成简单的几何形状,如网格,球体,圆柱体和盒子,我们在本书中将它们用于演示程序。 这个类在系统内存中生成数据,然后我们必须将我们想要的数据复制到顶点和索引缓冲区。 GeometryGenerator会创建一些将在后面的章节中使用的顶点数据。 我们当前的演示中不需要这些数据,因此我们不会将此数据复制到顶点缓冲区中。 MeshData结构是嵌套在GeometryGenerator中的简单结构,它存储顶点和索引列表:\\
\end{flushleft}
\begin{lstlisting}
class GeometryGenerator
{
public:
    using uint16 = std::uint16_t;
    using uint32 = std::uint32_t;

    struct Vertex
    {
        Vertex(){}
        Vertex(
            const DirectX::XMFLOAT3& p, 
            const DirectX::XMFLOAT3& n, 
            const DirectX::XMFLOAT3& t, 
            const DirectX::XMFLOAT2& uv) :
            Position(p), 
            Normal(n), 
            TangentU(t), 
            TexC(uv){}
        Vertex(
            float px, float py, float pz, 
            float nx, float ny, float nz,
            float tx, float ty, float tz,
            float u, float v) : 
            Position(px,py,pz), 
            Normal(nx,ny,nz),
            TangentU(tx, ty, tz), 
            TexC(u,v){}

        DirectX::XMFLOAT3 Position;
        DirectX::XMFLOAT3 Normal;
        DirectX::XMFLOAT3 TangentU;
        DirectX::XMFLOAT2 TexC;
    };

    struct MeshData
    {
        std::vector<Vertex> Vertices;
        std::vector<uint32> Indices32;

        std::vector<uint16>& GetIndices16()
        {
            if(mIndices16.empty())
            {
                mIndices16.resize(Indices32.size());
                for(size_t i = 0; i < Indices32.size(); ++i)
                    mIndices16[i] = static_cast<uint16>(Indices32[i]);
            }

            return mIndices16;
        }

    private:
        std::vector<uint16> mIndices16;
    };
......
};
\end{lstlisting}
%---- 7.4.1 ----
\subsection{创建圆柱体网格(Generating a Cylinder Mesh)}
\begin{flushleft}
我们通过指定圆柱体的底部和顶部半径,高度以及切片和堆叠数来定义圆柱体,如图\ref{fig:7-1}所示。 我们将圆柱体分成三个部分:1)侧面几何形状,2)顶盖几何形状,以及3)底盖几何形状。
\end{flushleft}
\begin{figure}[h]
    \includegraphics[width=\textwidth]{7-1}
    \centering
    \caption{在此图中,左侧的圆柱体有八个切片和四个堆叠,右侧的圆柱体有十六个切片和八个堆叠。 切片和堆栈控制三角形密度。 请注意,顶部和底部半径可以不同,以便我们可以创建锥形对象,而不仅仅是“纯”圆柱体。}
    \label{fig:7-1}
\end{figure}

%---- 7.4.1.1 ----
\subsubsection{圆柱体边缘几何(Cylinder Side Geometry)}
\begin{flushleft}
我们生成以原点为中心的圆柱体,平行于$y$轴。 从图\ref{fig:7-1}中,所有顶点都位于圆柱体的“环”上,其中有stackCount + 1个环,每个环都有 sliceCount 个(唯一)顶点。 连续环之间的半径差异为$\Delta r=(topRadius - bottomRadius)/ stackCount$。 如果我们从索引为0的底环开始,那么第i个环的半径是$r_{i} = bottomRadius + i\Delta r$,第i个环的高度是 $h_{i}=-\frac{h}{2}+i\Delta h$其中$\Delta h$是堆栈高度,h是圆柱高度。 因此,基本思想是迭代每个环,并生成位于该环上的顶点。 这给出了以下实现:\\
\end{flushleft}
\begin{lstlisting}
GeometryGenerator::MeshData 
GeometryGenerator::CreateCylinder(
    float bottomRadius, 
    float topRadius, 
    float height, 
    uint32 sliceCount, 
    uint32 stackCount)
{
    MeshData meshData;

    //
    // Build Stacks.
    // 

    float stackHeight = height / stackCount;

    // Amount to increment radius as we move up 
    // each stack level from bottom to top.
    float radiusStep = (topRadius - bottomRadius) / stackCount;

    uint32 ringCount = stackCount+1;

    // Compute vertices for each stack ring 
    // starting at the bottom and moving up.
    for(uint32 i = 0; i < ringCount; ++i)
    {
        float y = -0.5f*height + i*stackHeight;
        float r = bottomRadius + i*radiusStep;

        // vertices of ring
        float dTheta = 2.0f*XM_PI/sliceCount;
        for(uint32 j = 0; j <= sliceCount; ++j)
        {
            Vertex vertex;

            float c = cosf(j*dTheta);
            float s = sinf(j*dTheta);

            vertex.Position = XMFLOAT3(r*c, y, r*s);

            vertex.TexC.x = (float)j/sliceCount;
            vertex.TexC.y = 1.0f - (float)i/stackCount;

            // Cylinder can be parameterized as follows, where we introduce v
            // parameter that goes in the same direction as the v tex-coord
            // so that the bitangent goes in the same direction 
            // as the v tex-coord.
            //   Let r0 be the bottom radius and let r1 be the top radius.
            //   y(v) = h - hv for v in [0,1].
            //   r(v) = r1 + (r0-r1)v
            //
            //   x(t, v) = r(v)*cos(t)
            //   y(t, v) = h - hv
            //   z(t, v) = r(v)*sin(t)
            // 
            //  dx/dt = -r(v)*sin(t)
            //  dy/dt = 0
            //  dz/dt = +r(v)*cos(t)
            //
            //  dx/dv = (r0-r1)*cos(t)
            //  dy/dv = -h
            //  dz/dv = (r0-r1)*sin(t)

            // This is unit length.
            vertex.TangentU = XMFLOAT3(-s, 0.0f, c);

            float dr = bottomRadius-topRadius;
            XMFLOAT3 bitangent(dr*c, -height, dr*s);

            XMVECTOR T = XMLoadFloat3(&vertex.TangentU);
            XMVECTOR B = XMLoadFloat3(&bitangent);
            XMVECTOR N = XMVector3Normalize(XMVector3Cross(T, B));
            XMStoreFloat3(&vertex.Normal, N);

            meshData.Vertices.push_back(vertex);
        }
    }
......
\end{lstlisting}
\begin{flushleft}
NOTICE: 观察到每个环的第一个和最后一个顶点在位置上重复,但纹理坐标不重复。 我们必须这样做,以便我们可以正确地将纹理应用于圆柱体。\\
~\\
NOTICE: 实际上GeometryGenerator::CreateCylinder创建了额外的顶点数据,例如法线向量和纹理坐标,这些对未来的演示非常有用。 现在不要担心这些数量。\\
~\\
从图\ref{fig:7-2}可以看出,每个堆栈中的每个切片都有一个四边形(两个三角形)。 图\ref{fig:7-2}显示第$i$个堆栈和第$j$个切片的索引由下式给出:\\
\end{flushleft}
\begin{align*}
\Delta ABC = (i\cdot n+j, (i+1)\cdot n+j,(i+1)\cdot n+j+1)\\
\Delta ACD = (i\cdot n+j, (i+1)\cdot n+j+1,i\cdot n+j+1)
\end{lstlisting}
\begin{align*}
其中n是每个环的顶点数。 所以关键的想法是遍历每个堆栈中的每个切片,并应用上面的公式。\\
\end{flushleft}
\begin{lstlisting}
    // Add one because we duplicate the first and last vertex per ring
    // since the texture coordinates are different.
    uint32 ringVertexCount = sliceCount+1;

    // Compute indices for each stack.
    for(uint32 i = 0; i < stackCount; ++i)
    {
        for(uint32 j = 0; j < sliceCount; ++j)
        {
            meshData.Indices32.push_back(i*ringVertexCount + j);
            meshData.Indices32.push_back((i+1)*ringVertexCount + j);
            meshData.Indices32.push_back((i+1)*ringVertexCount + j+1);

            meshData.Indices32.push_back(i*ringVertexCount + j);
            meshData.Indices32.push_back((i+1)*ringVertexCount + j+1);
            meshData.Indices32.push_back(i*ringVertexCount + j+1);
        }
    }

    BuildCylinderTopCap(bottomRadius, topRadius, height, sliceCount, stackCount, meshData);
    BuildCylinderBottomCap(bottomRadius, topRadius, height, sliceCount, stackCount, meshData);

    return meshData;
}
\end{lstlisting}
\begin{figure}[h]
    \includegraphics[width=\textwidth]{7-2}
    \centering
    \caption{包含在第$i$个和第$i + 1$个环中的顶点A,B,C,D和第$j$个切片。}
    \label{fig:7-2}
\end{figure}

%---- 7.4.1.2 ----
\subsubsection{顶面底面几何(Cap Geometry)}
\begin{flushleft}
生成顶面几何图形相当于生成顶部和底部环的切片三角形以近似圆形:\\
\end{flushleft}
\begin{lstlisting}
void GeometryGenerator::BuildCylinderTopCap(
    float bottomRadius, 
    float topRadius, 
    float height,
    uint32 sliceCount, 
    uint32 stackCount, 
    MeshData& meshData)
{
    uint32 baseIndex = (uint32)meshData.Vertices.size();

    float y = 0.5f*height;
    float dTheta = 2.0f*XM_PI/sliceCount;

    // Duplicate cap ring vertices because 
    // the texture coordinates and normals differ.
    for(uint32 i = 0; i <= sliceCount; ++i)
    {
        float x = topRadius*cosf(i*dTheta);
        float z = topRadius*sinf(i*dTheta);

        // Scale down by the height to try and 
        // make top cap texture coord area
        // proportional to base.
        float u = x/height + 0.5f;
        float v = z/height + 0.5f;

        meshData.Vertices.push_back(
            Vertex(x, y, z, 
                0.0f, 1.0f, 0.0f, 
                1.0f, 0.0f, 0.0f, 
                u, v));
    }

    // Cap center vertex.
    meshData.Vertices.push_back(
        Vertex(0.0f, y, 0.0f, 
            0.0f, 1.0f, 0.0f, 
            1.0f, 0.0f, 0.0f, 
            0.5f, 0.5f));

    // Index of center vertex.
    uint32 centerIndex = (uint32)meshData.Vertices.size()-1;

    for(uint32 i = 0; i < sliceCount; ++i)
    {
        meshData.Indices32.push_back(centerIndex);
        meshData.Indices32.push_back(baseIndex + i+1);
        meshData.Indices32.push_back(baseIndex + i);
    }
}
\end{lstlisting}
\begin{flushleft}
底面类似。\\
\end{flushleft}

