\chapter{渲染管道(The Rendering Pipeline)}
\section{3D图像(The 3D Illusion)}
\section{模型表达(Model Representation)}
\section{基本计算颜色(Basic Computer Color)}
\begin{flushleft}
计算机显示器在每个像素发射一种混合红、绿、蓝三种颜色的光。当这种混合光到达眼睛并且击中视网膜区域,视锥细胞收到刺激并将产生的神经冲动通过视觉神经传递到大脑。大脑解释这种信号为颜色。随着混合光的变化,使这些视锥细胞收到不同的刺激,从而使大脑中产生不同的颜色。\\
\end{flushleft}

\subsection{颜色运算(Color Operations)}
\begin{itemize}
    \item 加法:$(0.0, 0.5, 0) + (0, 0.0, 0.25) = (0.0, 0.5, 0.25)$
    \item 减法:$(1, 1, 1) - (1, 1, 0) = (0, 0, 1)$
    \item 标量乘法:$0.5(1, 1, 1) = (0.5, 0.5, 0.5)$
    \item 显然,点乘和叉乘对于颜色向量来说没有意义。然而,颜色向量有一特殊的运算称为调制或分量乘法:$(c_{r},c_{g},c_{b}) \otimes (k_{r},k_{g},k_{b}) = (c_{r}k_{r},c_{g}k_{g},c_{b}k_{b})$。该运算主要用来作为照明方程。举个例子,假设我们有一束入射光线(r,g,b),它会照射一个反射50%红光,75%绿光和25%蓝光的表面,并吸收剩余的光线。 然后反射光线的颜色由下式给出:$(r,g,b) \otimes (0.5,0.75,0.25) = (0.5r,0.75g,0.25b)$
\end{itemize}

\subsection{128位颜色(128-Bit Color)}
\begin{flushleft}
通常我们会合并另外的颜色分量,称作 alpha 分量。alpha 分量常用来表示颜色的透明度(透明度在混合中很有用,因为我们还没有使用混合,目前只需将 alpha 分量设为 1 即可)。包含 alpha 分量意味着我们用 4D 颜色向量$(r, g, b, a), 0 \leq r,g,b,a \leq 1$ 为了用 128-bits 表示一种颜色,每个分量使用浮点型值。因为数学上,一种颜色就是一个 4D 向量,我们能在代码中使用 XMVECTOR 类型来表示一种颜色,并且每当调用 DirectX Math 中的向量函数时都会受益于 SIMD (如,颜色加减和标量乘法)。为了方便分量乘法,DirectX Math 库提供了下面方法:
\begin{lstlisting}
// Return c1 \otimes c2
XMVECTOR XM_CALLCONV XMColorModulate(FXMVECTOR C1, FXMVECTOR C2);
\end{lstlisting}
\end{flushleft}

\subsection{32位颜色(32-Bit Color)}
\begin{flushleft}
要以32位大小来表示一个颜色,可以给每个分量分配一个字节(8-bit)大小。这样一来,每个分量最多能表现256种色调——0代表无强度,255代表满强度,中间值代表中间强度。看似每个分量一个字节很小,但是组合在一起$(256 \times 256 \times 256)$就表示百万种不同的颜色。DirectX Math 库(\#include <DirectXPackedVector.h>) DirectX::PackedVector 命名空间(namespace)下提供了以下数据结构来存储32位颜色:
\begin{lstlisting}
namespace DirectX
{
namespace PackedVector
{
    // ARGB Color; 8-8-8-8 bit unsigned normalized integer components packed
    // into a 32 bit integer. The normalized color is packed into 32 bits
    // using 8 bit unsigned, normalized integers for the alpha, red, green,
    // and blue components.
    // The alpha component is stored in the most significant bits and the
    // blue component in the least significant bits (A8R8G8B8):
    // [32] aaaaaaaa rrrrrrrr gggggggg bbbbbbbb [0]
    struct XMCOLOR
    {
        union
        {
            struct
            {
                uint8_t b; // Blue: 0/255 to 255/255
                uint8_t g; // Green: 0/255 to 255/255
                uint8_t r; // Red: 0/255 to 255/255
                uint8_t a; // Alpha: 0/255 to 255/255
            };
            uint32_t c;
        };
        XMCOLOR() {}
        XMCOLOR(uint32_t Color) : c(Color) {}
        XMCOLOR(float _r, float _g, float _b, float _a);
        explicit XMCOLOR(_In_reads_(4) const float *pArray);
        operator uint32_t () const { return c; }
        XMCOLOR& operator= (const XMCOLOR& Color) 
        { c = Color.c; return *this; }
        XMCOLOR& operator= (const uint32_t Color) 
        { c = Color; return *this; }
    };
} // end PackedVector namespace
} // end DirectX namespace
\end{lstlisting}
一个32位颜色数据能被转换为128位颜色数据: 将整数区间$[0, 255]$ 映射到实数区间$[0, 1]$。每个数除以255即可,也就是说 设$0 \leq n \leq 255$,且 n 是整数,则 $0 \leq \frac{n}{255} \leq 1$ 就是需要的范围 0 到 1 的颜色强度。例:(80,140,200,255)转换如下:
$$ (80,140,200,255)\rightarrow (\frac{80}{255},\frac{140}{255},\frac{200}{255},\frac{255}{255}) \approx (0.31,0.55,0.78,1.0)$$
另一方面,128位颜色能转换为32位颜色:将每个分量乘以255并且四舍五入取整。例:
$$(0.3,0.6,0.9,1.0)\rightarrow (0.3\cdot 255,0.6\cdot 255,0.9\cdot 255,1.0\cdot 255)=(77,153,230,255)$$

在将32位颜色转换为128位颜色时通常必须执行额外的位操作,反之,因为8位颜色组件通常打包为32位整数值(如 unsigned int),XMCOLOR 就是如此。DirectXMath 库定义了一个方法以 XMCOLOR 作为参数,返回XMVECTOR:
\begin{lstlisting}
XMVECTOR XM_CALLCONV PackedVector::XMLoadColor(const XMCOLOR* pSource);
\end{lstlisting}
\end{flushleft}

\section{渲染管道概览(Overview of the rendering pipeline)}
\section{输入汇编程序阶段(The Input Assembler Stage)}
\subsection{顶点(Vertices)}
\subsection{原始拓扑(Primitive Topology)}
\subsection{指数(Indices)}

\section{顶点着色器阶段(The Vertex Shader Stage)}
\begin{flushleft}
在整合(assembled)基本数据(primitives)之后,将顶点送到顶点着色器阶段。 顶点着色器可以被认为是输入顶点并输出顶点的函数。 绘制的每个顶点都将通过顶点着色器进行抽取; 事实上,我们可以在概念上考虑硬件上发生的以下情况:
\end{flushleft}
\begin{lstlisting}
for (UINT i = 0; i < numVertices; ++i)
    outputVertex[i] = VertexShader(inputVertex[i]);
\end{lstlisting}
\begin{flushleft}
顶点着色器函数是我们实现的,但它由GPU为每个顶点执行,因此它非常快。\\
许多特殊效果可以在顶点着色器中执行,例如变换,光照和置换贴图。 请记住,我们不仅可以访问输入顶点数据,还可以访问存储在GPU内存中的纹理和其他数据,例如转换矩阵和场景灯。\\
我们将在本书中看到许多不同顶点着色器的例子; 所以到最后,你应该知道可以用它们做些什么。 但是,对于我们的第一个代码示例,我们将使用顶点着色器来变换顶点。 以下小节解释了通常需要完成的转换类型。\\
\end{flushleft}

\subsection{局部空间和世界空间(Local Space and World Space)}
\begin{flushleft}
假设您正在制作一部电影并且您的团队必须为一些特殊效果镜头构建一个微型版本的火车场景。 特别是,假设你的任务是建造一座小桥。 现在,你不会在场景的中间构建桥梁,在那里你可能需要从一个困难的角度工作,并注意不要弄乱组成场景的其他微缩模型。 相反,您将在远离场景的工作台上工作。 然后,当完成所有操作后,您可以将桥放置在场景中的正确位置和角度。\\
3D艺术家在构建3D对象时会做类似的事情。它们不是用相对于全局场景坐标系(世界空间)的坐标构建对象的几何,而是相对于局部坐标系(局部空间)指定它们;局部坐标系通常是位于物体附近并与物体轴对齐的一些方便的坐标系。一旦在局部空间中定义了3D模型的顶点,就将其放置在全局场景中。为了做到这一点,我们必须确定当地空间和世界空间的相关性;这是通过指定我们想要局部空间坐标系的原点和轴相对于全局场景坐标系的位置,并执行坐标转换的更改来完成的(参见图\ref{fig:5-16}并回顾3.4节)。将相对于局部坐标系的坐标改变为全局场景坐标系的过程称为世界变换,相应的矩阵称为世界矩阵。场景中的每个对象都有自己的世界矩阵。在每个对象从其局部空间转换到世界空间之后,所有对象的所有坐标都相对于同一坐标系(世界空间)。如果要直接在世界空间中定义对象,则可以提供标识世界矩阵。\\
\end{flushleft}
\begin{figure}[t]
    \includegraphics[width=\textwidth]{5-16}
    \centering
    \caption{(a)每个对象的顶点用相对于它们自己的局部坐标系的坐标定义。 另外,我们根据我们想要场景中的对象来定义每个局部坐标系相对于世界空间坐标系的位置和方向。 然后我们执行坐标变换的更改,以使所有坐标相对于世界空间系统。 (b)在世界变换之后,对象的顶点具有相对于同一世界系统的所有坐标。}
    \label{fig:5-16}
\end{figure}

\begin{figure}[t]
    \includegraphics[width=\textwidth]{5-17}
    \centering
    \caption{当立方体在原点居中并与坐标系轴对齐时,可以轻松指定立方体的顶点。 当立方体相对于坐标系处于任意位置和方向时,指定坐标并不容易。 因此,当我们构造一个对象的几何体时,我们通常总是在对象附近选择一个方便的坐标系并与对象对齐,从中构建对象。}
    \label{fig:5-17}
\end{figure}
\begin{flushleft}
相对于自己的局部坐标系定义每个模型有几个优点:\\
\end{flushleft}
\begin{itemize}
  \item 1. 更方便。例如,通常在局部空间中,对象将以原点为中心并相对于一个主轴对称。 另一个例子,如果我们选择一个原点位于立方体并且轴与立方体面正交的局部坐标系,则立方体的顶点更容易指定; 见图\ref{fig:5-17}。
  \item 2. 该对象可以在多个场景中重复使用,在这种情况下,相对于特定场景硬编码对象的坐标是没有意义的。 相反,最好相对于局部坐标系存储其坐标,然后通过坐标矩阵的变化来定义局部坐标系和世界坐标系如何与每个场景相关。
  \item 3. 最后,有时我们在场景中不止一次地绘制相同的对象,但是在不同的位置,方向和比例(例如,树对象可以多次重复使用以构建森林)。 复制每个实例的对象的顶点和索引数据将是浪费的。 相反,我们存储相对于其局部空间的几何(即顶点和索引列表)的单个副本。 然后我们多次绘制对象,但每次都使用不同的世界矩阵来指定实例在世界空间中的位置,方向和比例。 这称为实例化。
\end{itemize}
\begin{flushleft}
如3.4.3节所示,对象的世界矩阵是通过用相对于世界空间的坐标描述其局部空间,并将这些坐标放在矩阵的行中给出的。如果$Q_{w}=(Q_{x},Q_{y},Q_{z},1)$,则$u_{w}=(u_{x},u_{y},u_{z},0)$,$v_{w}=(v_{x},v_{y},v_{z},0)$,并且$w_{w}=(w_{x},w_{y},w_{z},0)$分别描述了具有相对于世界空间的齐次坐标的局部空间的原点,x轴,y轴和z轴,然后我们从3.4.3节中知道坐标矩阵从局部空间到世界空间变化是:\\
\end{flushleft}
\begin{align*}
W=
\begin{bmatrix}
u_{x} & u_{y} & u_{z} & 0\\
v_{x} & v_{y} & v_{z} & 0\\
w_{x} & w_{y} & w_{z} & 0\\
Q_{x} & Q_{y} & Q_{z} & 1
\end{bmatrix}\\
\end{align*}
\begin{flushleft}
我们看到,要构造一个世界矩阵,我们必须直接计算出局部空间原点的坐标和相对于世界空间的轴。 这有时不那么容易或直观。更常见的方法是将 $W$ 定义为变换序列,例如 $W=SRT$,缩放矩阵 $S$ 的乘积,用于将对象缩放到世界中,接着是旋转矩阵 $R$,以定义局部空间相对于的方向。 世界空间,后面是一个平移矩阵 $T$,用于定义相对于世界空间的局部空间的起源。 从3.5节开始,我们知道这个变换序列可以解释为坐标变换的变化,并且 $W=SRT$ 的行向量存储$x$轴,$y$轴,$z$轴和原点的齐次坐标。相对于世界空间的局部空间。
\end{flushleft}

\begin{flushleft}
例子:\\
假设我们有一个相对于某个局部空间定义的单位方形平面,分别具有最小和最大点$(-0.5,0,-0.5)$和$(0.5,0,0.5)$。 找到世界矩阵,使得正方形在世界空间中的长度为$2$,正方形在世界空间的$xz$平面中顺时针旋转$45^{\circ}$,并且正方形位于世界空间的 $(10,0,10)$ 处。 我们按如下方式构造 $S$,$R$,$T$和$W$:\\
\end{flushleft}
\begin{align*}
S=
\begin{bmatrix}
2 & 0 & 0 & 0\\
0 & 1 & 0 & 0\\
0 & 0 & 2 & 0\\
0 & 0 & 0 & 1
\end{bmatrix}
R=
\begin{bmatrix}
\sqrt{2}/2 & 0 & -\sqrt{2}/2 & 0\\
0 & 1 & 0 & 0\\
\sqrt{2}/2 & 0 & \sqrt{2}/2 & 0\\
0 & 0 & 0 & 1
\end{bmatrix}
T=
\begin{bmatrix}
1 & 0 & 0 & 0\\
0 & 1 & 0 & 0\\
0 & 0 & 1 & 0\\
10 & 0 & 10 & 1
\end{bmatrix}\\
W=SRT=
\begin{bmatrix}
\sqrt{2} & 0 & -\sqrt{2} & 0\\
0 & 1 & 0 & 0\\
\sqrt{2} & 0 & \sqrt{2} & 0\\
10 & 0 & 10 & 1
\end{bmatrix}
\end{align*}

\begin{flushleft}
现在从3.5节开始,$W$ 中的行描述了相对于世界空间的局部坐标系; 也就是说,$u_{w}=(\sqrt{2},0,-\sqrt{2},0),v_{w}=(0,1,0,0),w_{w}=(\sqrt{2},0,\sqrt{2},0)$,并且 $Q_{w}=(10,0,10,1)$。 当我们用 $W$ 将局部空间转变为世界空间的坐标时,方形平面也就位于世界空间中的所需位置了(见图\ref{fig:5-18})。
\end{flushleft}

\begin{align*}
[-0.5,0,-0.5,1]W=[10-\sqrt{2},0,0,1]\\
[-0.5,0,+0.5,1]W=[0,0,10+\sqrt{2},1]\\
[+0.5,0,+0.5,1]W=[10+\sqrt{2},0,0,1]\\
[+0.5,0,-0.5,1]W=[0,0,10-\sqrt{2},1]\\
\end{align*}
\begin{flushleft}
这个例子的要点是,我们不是直接计算$Q_{w}$,$u_{w}$,$v_{w}$ 和 $w_{w}$来形成世界矩阵,而是通过合成一系列简单变换来构造世界矩阵。 这通常比直接计算$Q_{w}$,$u_{w}$,$v_{w}$ 和 $w_{w}$容易得多,因为我们只需要问:我们希望物体在世界空间中的大小,我们希望物体在世界空间中的方向, 我们在什么位置想要世界空间中的物体。
\end{flushleft}

\begin{figure}[t]
    \includegraphics[width=\textwidth]{5-18}
    \centering
    \caption{世界矩阵的行向量描述了具有相对于世界坐标系的坐标的局部坐标系。}
    \label{fig:5-18}
\end{figure}

\begin{flushleft}
考虑世界变换的另一种方法是仅采用局部空间坐标并将它们视为世界空间坐标(这相当于使用单位矩阵作为世界变换)。 因此,如果对象在其局部空间的中心建模,则该对象恰好位于世界空间的中心。 一般来说,世界的中心可能不是我们想要定位所有物体的地方。 所以现在,对于每个对象,只需应用一系列变换来缩放,旋转和定位您想要在世界空间中的对象。 在数学上,这将给出与从局部空间到世界空间的坐标矩阵变化相同的世界变换。
\end{flushleft}

\subsection{视图空间(View Space)}
\begin{flushleft}
为了形成场景的二维图像,我们必须在场景中放置一个虚拟相机。 相机指定了观众可以看到的世界的多大的体积,以及我们需要生成多大的世界体积来生成二维图像。让我们将一个局部坐标系统(称为视图空间,眼图空间或相机空间)附加到相机,如图\ref{fig:5-19}所示; 也就是说,摄像机坐落在正视z轴的原点处,x轴指向摄像机的右侧,y轴指向摄像机的上方。 而不是描述相对于世界空间的场景顶点,基于相机坐标系,渲染管线的后面阶段很方便地描述它们。从世界空间到视图空间的坐标变换称为视图变换,相应的矩阵称为视图矩阵。\\
\begin{figure}[t]
    \includegraphics[width=\textwidth]{5-19}
    \centering
    \caption{转换顶点相对于世界空间的坐标,使它们相对于相机空间。}
    \label{fig:5-19}
\end{figure}
如果$Q_{w}=(Q_{x},Q_{y},Q_{z},1)$,$u_{w}=(u_{x},u_{y},u_{z},0)$,$v_{w}=(v_{x},v_{y},v_{z},0)$,$w_{w}=(w_{x},w_{y},w_{z},0)$分别用相对于世界空间的齐次坐标来描述视图空间的原点,x轴,y轴和z轴,然后我们从3.4.3一节知道坐标矩阵从视图空间的变化到世界空间是:
$$W=
\begin{bmatrix}
u_{x} & u_{y} & u_{z} & 0\\
v_{x} & v_{y} & v_{z} & 0\\
w_{x} & w_{y} & w_{z} & 0\\
Q_{x} & Q_{y} & Q_{z} & 1
\end{bmatrix}$$\\
然而,这不是我们想要的转换。我们想要的是倒置转换,即从世界空间转换成视图空间。回忆3.4.3节,倒置转换就是求矩阵的逆。因此$W^{-1}$就是从世界空间到视图空间的转换。\\
世界坐标系统和视图坐标系统仅在位置和方向有区别,所以直观地得到 $W=RT$(也就是说世界矩阵能被分解为一个旋转跟一个平移(translation))\footnote{$R$是\href{https://en.wikipedia.org/wiki/Orthogonal_matrix}{\textcolor{linkColor}{正交矩阵}},所以有 $R^{-1}=R^{T}$}。这让求逆变得简单:
\begin{align*}
V&=W^{-1}=(RT)^{-1}=T^{-1}R^{-1}=T^{-1}R^{T} \\
&=\begin{bmatrix}
1 & 0 & 0 & 0\\
0 & 1 & 0 & 0\\
0 & 0 & 1 & 0\\
-Q_{x} & -Q_{y} & -Q_{z} & 1
\end{bmatrix}
\begin{bmatrix}
u_{x} & v_{x} & w_{x} & 0\\
u_{y} & v_{y} & w_{y} & 0\\
u_{z} & v_{z} & w_{z} & 0\\
0 & 0 & 0 & 0
\end{bmatrix} \\
&=\begin{bmatrix}
u_{x} & v_{x} & w_{x} & 0\\
u_{y} & v_{y} & w_{y} & 0\\
u_{z} & v_{z} & w_{z} & 0\\
\mathbf{-Q\cdot u} & \mathbf{-Q\cdot v} & \mathbf{-Q\cdot w} & 1
\end{bmatrix}
\end{align*}

所以,视图矩阵就是:
\begin{align*}
V=\begin{bmatrix}
u_{x} & v_{x} & w_{x} & 0\\
u_{y} & v_{y} & w_{y} & 0\\
u_{z} & v_{z} & w_{z} & 0\\
\mathbf{-Q\cdot u} & \mathbf{-Q\cdot v} & \mathbf{-Q\cdot w} & 1
\end{bmatrix}
\end{align*}
现在我们用一种直观的方式来构造向量,这些向量用来建立视图矩阵。令\textbf{Q}为摄像机的位置,\textbf{T}为摄像机对准的目标点。然后,令\textbf{j}为世界空间向上方向的单位向量。(本书中,世界坐标系的 xz 构成的平面作为世界地平面,世界 y 轴描述了向上的方向;所以,$j=(0,1,0)$ 就是和世界 y 轴平行的单位向量。然而这只是约定,一些应用可能会选择 xy构成的平面作为地平面,z轴作为向上方向。)参考图片\ref{fig:5-20},摄像机的朝向如下:
$$w=\frac{T-Q}{||T-Q||}$$
该向量描述的是摄像机的z轴,对准\textbf{w}右侧的单位向量:\\
$$u=\frac{j\times w}{||j \times w||}$$
该向量描述的是摄像机的 x 轴。最后,摄像机的y轴单位向量为:\\
$$v=w \times u$$
因为\textbf{w}和\textbf{u}是互相正交的单位向量,$w\times u$也必定是单位向量,没必要正规化。\\
综上所述,给定摄像机的位置,目标点,和世界向上方向,我们就能求得摄像机的本地坐标系统,作为视图矩阵。
\begin{figure}[t]
    \includegraphics[width=\textwidth]{5-20}
    \centering
    \caption{根据相机位置,目标点和世界“向上”矢量构造相机坐标系。}
    \label{fig:5-20}
\end{figure}

DirectXMath 库提供了计算视图矩阵的方法:
\begin{lstlisting}
// Outputs view matrix V
XMVECTOR XM_CALLCONV XMMatrixLookAtLH(
    FXMVECTOR EyePosition,    // Input camera position Q
    FXMVECTOR FocusPosition,  // Input target point T
    FXMVECTOR UpDirection);   // Input world up direction j
\end{lstlisting}
通常世界y轴就是向上的方向,所以$j=(0,1,0)$ 就是向上的向量。例:假设我们将摄像机放在相对于世界坐标的$(5,3,-10)$处,并且对准世界原点坐标$(0,0,0)$。可以通过下面方式获得视图矩阵:\\
\begin{lstlisting}
XMVECTOR pos    = XMVectorSet(5, 3, -10, 1.0f);
XMVECTOR target = XMVectorZero();
XMVECTOR up     = XMVectorSet(0.0f, 1.0f, 0.0f, 0.0f);
XMMATRIX V      = XMMatrixLookAtLH(pos, target, up);
\end{lstlisting}
\end{flushleft}
\subsection{投影和齐次裁剪空间(Projection and Homogeneous Clip Space)}
\begin{flushleft}
到目前为止我们描述了在世界空间中摄像机的位置和方向,但摄像机还有一个部分需要解决,即摄像机看到的空间体积。体积由视锥来描述(见图\ref{fig:5-21})。
\end{flushleft}
\begin{figure}[t]
    \includegraphics[width=\textwidth]{5-21}
    \centering
    \caption{视锥定义了摄像机“看到”的空间体积}
    \label{fig:5-21}
\end{figure}
\begin{flushleft}
我们接下来的任务是将一个 3D 几何体在视锥中投影到 2D 投影窗口中。投影必须以平行线汇聚到消失点的方式完成,当一个物体的3D深度增加(距离拉远),投影的大小就减小。这就是\href{https://en.wikipedia.org/wiki/Perspective_(graphical)}{\textcolor{linkColor}{透视投影}}(如图\ref{fig:5-22})。我们将一个顶点到视点(眼睛所在的位置,姑且叫视点吧)的连线称为顶点的投影线。然后我们定义透视投影变换就是3D顶点\textbf{v}变换成它的投影线和2D投影面板交叉点\textbf{v'}的过程;我们认为\textbf{v'}就是\textbf{v}的投影。一个3D物体的投影就是组成这个物体的所有顶点的投影。
\end{flushleft}
\begin{figure}[t]
    \includegraphics[width=\textwidth]{5-22}
    \centering
    \caption{两个圆柱体大小相同,但深度不同。距离眼睛近的圆柱体比距离远的圆柱体投影大。视锥中的几何体被投影到投影窗口;视锥外的集合体,投影到投影面板上,但在投影窗口外面(投影面板和投影窗口共面)}
    \label{fig:5-22}
\end{figure}

\paragraph{定义一个视锥(Defining a Frustum)}
\begin{flushleft}
我们可以在视图空间中定义一个视锥,其投影中心位于原点处,并沿着正z轴向下看,通过以下四个量:近平面$n$,远平面$f$,垂直视场角$\alpha$ 和纵横比 $r$。请注意,在视图空间中,近平面和远平面平行于$xy$平面; 因此我们只需指定它们沿着z轴的原点距离。 纵横比由 $r=w/h$ 定义,其中$w$是投影窗口的宽度,$h$是投影窗口的高度(视图空间中的单位)。 投影窗口本质上是视图空间中场景的二维图像。 这里的图像最终将映射到后台缓冲区(back buffer); 因此,我们喜欢投影窗口尺寸的比率与后台缓冲区尺寸的比率相同。所以后缓冲区维度的比率通常被指定为长宽比(这是一个比例,所以它没有单位)。例,如果后台缓冲区尺寸是 $800 \times 600$,则 $r=\frac{800}{600} \approx 1.333$。如果投影窗口与后台缓冲区的纵横比不相同,则需要非均匀缩放来将投影窗口映射到后台缓冲区,这将导致失真(例如,投影窗口上的圆圈可能当映射到后台缓冲区时被拉伸成一个椭圆)。\\
我们将水平视角标记为$\beta$,垂直视角标记为$\alpha$,纵横比为$r$。要看看$r$如何帮助我们找到$\beta$,请考虑图\ref{fig:5-23}。 请注意,投影窗口的实际尺寸并不重要,只需要保持纵横比。 因此,我们会选择2的方便高度,因此宽度必须为:\\
$$r=\frac{w}{h}=\frac{w}{2}\Rightarrow w=2r$$
\end{flushleft}
\begin{figure}[t]
    \includegraphics[width=\textwidth]{5-23}
    \centering
    \caption{给定垂直视角$\alpha$和纵横比$r$,得出水平视角$\beta$}
    \label{fig:5-23}
\end{figure}
\begin{flushleft}
为了具有指定的垂直视场$\alpha$,投影窗口必须放置在距离原点的距离$d$处:
$$\tan(\frac{\alpha}{2})=\frac{1}{d}\Rightarrow d=\cot(\frac{\alpha}{2})$$
现在我们已经将投影窗沿$z$轴的距离d固定为当投影窗的高度为2时垂直视场$\alpha$。现在我们可以求解$\beta$。 看一下图\ref{fig:5-23}中的$xz$平面,我们现在看到:
$$\tan(\frac{\beta}{2})=\frac{r}{d}=\frac{r}{\cot(\frac{\alpha}{2})}=r\cdot \tan(\frac{\alpha}{2})$$
因此,考虑到垂直视角α和纵横比r,我们总能得到水平视角β:
$$\beta=2\tan^{-1}(r\cdot\tan(\frac{\alpha}{2}))$$
\end{flushleft}

\paragraph{投影顶点(Projecting Vertices)}
\begin{figure}[t]
    \includegraphics[width=\textwidth]{5-24}
    \centering
    \caption{相似三角}
    \label{fig:5-24}
\end{figure}
\begin{flushleft}
参考图\ref{fig:5-24}。 给定一个点$(x,y,z)$,我们希望在投影平面$z = d$上找到它的投影$(x',y',d)$。 通过分别考虑$x$和$y$坐标并使用相似的三角形,我们发现:
\begin{align*}
\frac{x'}{d}=\frac{x}{z}\Rightarrow x'=\frac{xd}{z}=\frac{x\cot(\alpha/2)}{z}=\frac{x}{z\tan(\alpha/2)}\\
\frac{y'}{d}=\frac{y}{z}\Rightarrow y'=\frac{yd}{z}=\frac{y\cot(\alpha/2)}{z}=\frac{y}{z\tan(\alpha/2)}\\
\end{align*}
显然对于点$(x,y,z)$, $-r\leq x'\leq r,-1\leq y'\leq 1,n\leq z\leq f$
\end{flushleft}

\paragraph{标准化的设备坐标(Normalized Device Coordinate-NDC)}
\begin{flushleft}
上一节中投影点的坐标是在视图空间中计算的。 在视图空间中,投影窗口的高度为2,宽度为$2r$,其中$r$是纵横比。 问题在于,尺寸取决于宽高比。 这意味着我们需要告诉硬件高宽比,因为硬件稍后需要做一些涉及投影窗口尺寸的操作(比如将其映射到后台缓冲区)。 如果我们能够消除这种对长宽比的依赖性会更方便。 解决的办法是将投影的$x$坐标从区间$[-r,r]$缩放到$[-1,1]$像这样:
\begin{align*}
-r\leq x' \leq r \\
-1\leq x'/r \leq 1
\end{align*}

在映射之后,$x$坐标和$y$坐标被称为标准化的设备坐标(NDC)(z坐标还没有被标准化),并且点$(x,y,z)$在视锥内当且仅当:
\begin{align*}
-1\leq x'/r \leq 1 \\
-1\leq y' \leq 1 \\
n \leq z \leq f
\end{align*}

从视图空间到NDC空间的转换可视为单位转换。 我们有一个关系,即一个NDC单位在$x$轴上等于视图空间中的$r$个单位(即 1ndc = \textbf{r} vs)。 所以给定$x$个视图空间单位,我们可以使用这个关系来转换单位:
$$x\mathit{vs}\cdot \frac{1\mathit{ndc}}{r\mathit{vs}}=\frac{x}{r}\mathit{ndc}$$
我们可以修改我们的投影公式,直接在NDC坐标中为我们提供投影的$x$坐标和$y$坐标:
\begin{align*}\tag{eq.5.1}\label{eq.5.1}
x'&=\frac{x}{rz\tan(\alpha/2)}\\
y'&=\frac{y}{z\tan(\alpha/2)}
\end{align*}
请注意,在NDC坐标中,投影窗口的高度为2,宽度为2.因此,现在尺寸是固定的,硬件无需知道纵横比,但我们有责任始终在NDC中提供投影坐标空间(图形硬件假设我们会)。
\end{flushleft}

\paragraph{用矩阵来描述投影公式(Writing the Projection Equations with a Matrix)}
\begin{flushleft}
为了保持一致,我们将用一个矩阵来描述投影变换。不过,公式\ref{eq.5.1}是非线性的,无法用矩阵描述。所以我们要使用一种“技巧”将它分为两部分来实现:一个线性部分和一个非线性部分。非线性部分要除以$z$。我们会在下一节讨论“如何规范化$z$坐标”时讲解这一问题;现在读者只需要知道,我们会因为个除法操作而失去原始的z坐标。所以,我们必须在变换之前保存输入的z坐标;我们可以利用齐次坐标来解决一问题,将输入的$z$坐标复制给输出的$w$坐标。在矩阵乘法中,我们要将元素[2][3]设为1、元素[3][3]设为0(从0开始的索引)。我们的投影矩阵大致如下:
\begin{align*}
P=\begin{bmatrix}
\frac{1}{r\tan(\alpha/2)} & 0 & 0 & 0\\
0 & \frac{1}{\tan(\alpha/2)} & 0 & 0\\
0 & 0 & A & 1\\
0 & 0 & B & 0
\end{bmatrix}
\end{align*}
注意矩阵中的常量$A$和$B$(它们将在下一节讨论);这些常量用于把输入的$z$坐标变换到规范化区间。将一个任意点$(x,y,z,1)$与该矩阵相乘,可以得到:
\begin{align*}\tag{eq.5.2}\label{eq.5.2}
[x,y,z,1]\begin{bmatrix}
\frac{1}{r\tan(\alpha/2)} & 0 & 0 & 0\\
0 & \frac{1}{\tan(\alpha/2)} & 0 & 0\\
0 & 0 & A & 1\\
0 & 0 & B & 0
\end{bmatrix}=\left[ \frac{x}{r\tan(\alpha/2)},\frac{y}{\tan(\alpha/2)},Az+B,z\right]
\end{align*}
在与投影矩阵(线性部分)相乘之后,我们要将每个坐标除以$w = z$(非线性部分),得到最终的变换结果:
\begin{align*}\tag{eq.5.3}\label{eq.5.3}
\left[ \frac{x}{r\tan(\alpha/2)},\frac{y}{\tan(\alpha/2)},Az+B,z\right]
\xrightarrow[]{\text{除以$w$}}
\left[ \frac{x}{rz\tan(\alpha/2)},\frac{y}{z\tan(\alpha/2)},A+\frac{B}{z},1\right]
\end{align*}
顺便提一句,你可能会问:“如何处理除数为0的情况”;对于一问题我们不必担心,因为近平面总是大于0的,其他的点都会被裁剪掉(参见5.9节)。有时,与$w$相除的过程也称为透视除法(perspective divide)或齐次除法(homogeneous divide)。我们可以看到$x$、$y$的投影坐标与公式\ref{eq.5.1}相同。
\end{flushleft}

\paragraph{规范化深度值}
\begin{flushleft}
你可能认为在投影之后可以丢弃原始的3D $z$坐标,因为所有的投影点已经摆放在2D投影窗口上,形成了我们最终看到的2D图像,不会再使用3D $z$坐标了。其实不然,我们仍然需要为深度缓存算法提供3D深度信息。就如同Direct3D希望我们把$x$、$y$投影坐标映射到一个规范化区间一样,Direct3D也希望我们将深度坐标映射到一个规范化区间$[0,1]$中。所以,我们需要创建一个保序函数(order preserving function)$g(x)$把$[n,f]$区间映射到$[0,1]$区间。由于该函数是保序的,所以当$z_{1},z_{2}∈[n,f]$且$z_{1}<z_{2}$时,必有$g(z_{1})<g(z_{2})$。这样,即使深度值已经被变换过了,相对的深度关系还是会被完好无损地保留下来,我们依然可以在规范化区间中得到正确的深度测试结果,这就是我们要为深度缓存算法做的全部工作。
通过缩放和平移可以实现从$[n ,f]$到$[0,1]$的映射。但是,这种方式无法与我们当前的投影方程整合。我们可以从公式\ref{eq.5.3}中看到经过变换的$z$坐标为:
$$g(z)=A+\frac{B}{z}$$
我们现在需要让$A$和$B$满足以下条件:
\begin{itemize}
    \item 条件1:$g(n)=A+\frac{B}{n}$(近平面映射为0)
    \item 条件2:$g(f)=A+\frac{B}{f}$(远平面映射为1)
\end{itemize}
由条件1得到$B$的结果为:$B=-An$。把它代入条件2,得到$A$的结果为:
\begin{align*}
A+\frac{-An}{f}=1\\
\frac{Af-An}{f}=1\\
Af-An=f\\
A=\frac{f}{f-n}
\end{align*}
所以:
\begin{align*}
g(z)=\frac{f}{f-n}-\frac{nf}{(f-n)z}
\end{align*}
从$g(z)$的曲线图(图\ref{fig:5-25})中可以看出,它会限制增长的幅度(保序)而且是非线性的。从图中我们还可以看到,区间中的大部分取值落在近平面附近。因此,大多数深度值被映射到了一个很窄的取值范围内。这会导致深度缓冲区出现精度问题(由于所能表示的数值范围有限,计算机将无法识别变换后的深度值之间的微小差异)。通常的建议是让近平面和远平面尽可能接近,把深度的精度性问题减小到最低程度。
\end{flushleft}
\begin{figure}[b]
    \includegraphics[width=\textwidth]{5-25}
    \centering
    \caption{相对于不同近平面的$g(z)$曲线图}
    \label{fig:5-25}
\end{figure}
\begin{flushleft}
现在我们已经解出了A和B,我们可以确定出完整的透视投影矩阵:
\begin{align*}
P=\begin{bmatrix}
\frac{1}{r\tan(\alpha/2)} & 0 & 0 & 0\\
0 & \frac{1}{\tan(\alpha/2)} & 0 & 0\\
0 & 0 & \frac{f}{f-n} & 1\\
0 & 0 & \frac{-nf}{f-n} & 0
\end{bmatrix}
\end{align*}
在与投影矩阵相乘之后,进行透视除法之前,几何体所处的空间称为齐次裁剪空间(homogeneous clip space)或投影空间(projection space)。在透视除法之后,几何体所处的空间称为规范化设备空间(normalized device coordinates,简称NDC)。
\end{flushleft}

\paragraph{XMMatrixPerspectiveFovLH}
\begin{flushleft}
一个透视投影矩阵能由下面的 DirectX Math 函数生成:
\begin{lstlisting}
// Return the projection matrix
XMMATRIX XM_CALLCONV XMMatrixPerspectiveFovLH(
    float FovAngleY, // vertical field of view angle in radians
    float Aspecct,     // aspect ratio = width / height
    float NearZ,       // distance to near plane
    float FarZ);         // distance to far plane
\end{lstlisting}
下面代码展示了如何使用 XMMatrixPerspectiveFovLH。这里,我们定义垂直视角为$45^{\circ}$,近平面$z=1$,远平面$z=1000$(这些长度都在视图空间)
\begin{lstlisting}
XMMATRIX P= XMMatrixPerspectiveFovLH(0.25f*XM_PI, AspectRatio(),1.0f,1000.0f);
\end{lstlisting}
纵横比匹配窗口的纵横比:
\begin{lstlisting}
float D3DApp::AspectRatio() const
{
  return static_cast<float>(mClientWidth) / mClientHeight;
}
\end{lstlisting}
\end{flushleft}

\section{镶嵌阶段(曲面细分阶段)(The Tessellation Stages)}
\begin{flushleft}
镶嵌(Tessellation)是指通过添加三角形的方式对一个网格的三角形进行细分,这些新添加的三角形可以偏移到一个新的位置,让网格的细节更加丰富。如图\ref{fig:5-26}
\end{flushleft}
\begin{figure}[h]
    \includegraphics[width=\textwidth]{5-26}
    \centering
    \caption{左图是原网格,右图是镶嵌后的网格}
    \label{fig:5-26}
\end{figure}

\begin{flushleft}
下面是曲面细分的好处:
\begin{itemize}
    \item 我们可以实现细节层次(level-of-detail, LOD),使靠近相机的三角形通过细分产生更多细节,而那些远离相机的三角形则保持不变。通过这种方式,我们只需在需要细节的地方使用更多的三角形就可以了。
    \item 我们可以在内存中保存一个低细节(低细节意味着三角形数量少)的网格,但可以实时地添加额外的三角形,这样可以节省内存。
    \item 我们可以在一个低细节的网格上处理动画和物理效果,而只在渲染时才使用细分过的高细节网格。
\end{itemize}
曲面细分阶段是Direct3D 11中新添加的,这样我们就可以在GPU上对几何体进行细分了。而在Direct3D 11之前,如果你想要实现曲面细分,则必须在CPU上完成,经过细分的几何体还要发送到GPU用于渲染。然而,将新的几何体从CPU内存发送到显存是很慢的,而且还会增加CPU的负担。因此,在Direct3D 11出现之前,曲面细分的方法在实时图形中并不流行。Direct3D 11提供了一个可以完全在硬件上实现的曲面细分API。这样曲面细分就成为了一个非常有吸引力的技术了。曲面细分阶段是可选的(即在需要的时候才使用它)。我们要在第14章才会详细介绍曲面细分。
\end{flushleft}

\section{几何着色阶段(The Geometry Shader Stage)}
\begin{flushleft}
几何着色器阶段(geometry shader stage)是可选的,我们在第12章之前不会用到它,所以这里只做一个简短的概述。几何着色器以完整的图元作为输入数据。例如,当我们绘制三角形列表时,输入到几何着色器的数据是构成三角形的三个点。(注意,这三个点是从顶点着色器传递过来的。)几何着色器的主要优势是它可以创建或销毁几何体。例如,输入图元可以被扩展为一个或多个其他图元,或者几何着色器可以根据某些条件拒绝输出某些图元。这一点与顶点着色器有明显的不同:顶点着色器无法创建顶点,只要输入一个顶点,那么就必须输出一个顶点。几何着色器通常用于将一个点扩展为一个四边形,或者将一条线扩展为一个四边形。\\
~\\
我们可以在图5.11中看到一个“流输出(stream output)”箭头。也就是,几何着色器可以将顶点数据流输出到内存中的一个顶点缓冲区内,这些顶点可以在管线的随后阶段中渲染出来。这是一项高级技术,我们会在后面的章节中对它进行讨论。\\
~\\
NOTICE:顶点位置在离开几何着色器之前,必须被变换到齐次裁剪空间。
\end{flushleft}

\section{裁剪(Clipping)}
\begin{flushleft}
我们必须完全丢弃在平截头体之外的几何体,裁剪与平截头体边界相交的几何体,只留下平截头体内的部分;图 \ref{fig:5-27}以2D形式说明了一概念。
\begin{figure}[h]
    \includegraphics[width=\textwidth]{5-27}
    \centering
    \caption{(a)裁剪之前,(b)裁剪之后}
    \label{fig:5-27}
\end{figure}
我们可以将平截头体视为由6个平面界定的空间范围:顶、底、左、右、近、远平面。要裁剪与平截头体方向相反的多边形,其实就是逐个裁剪与每个平截头体平面方向相反的多边形,当裁剪一个与平面方向相反的多边形时(参见图\ref{fig:5-28}),我们将保留平面正半空间中的部分,而丢弃平面负半空间中的部分。对一个与平面方向相反的凸多边形进行裁剪,得到的结果仍然会是一个凸多边形。由于硬件会自动完成所有的裁剪工作,所以我们不在这里讲解具体的实现细节;有兴趣的读者可以参阅[Sutherland74],了解一下目前流行的Sutherland-Hodgeman裁剪算法。它基本思路是:求出平面与多边形边之间的交点,然后对顶点进行排序,形成新的裁剪后的多边形。\\
\begin{figure}[h]
    \includegraphics[width=\textwidth]{5-28}
    \centering
    \caption{(a)裁剪一个与平面方向相反的三角形,(b)裁剪后的三角形。注意,裁剪后的三角形已经不再是一个三角形了,它是一个四边形。所以,硬件必须将这个四边形重新划分为三角形,对于凸多边形来说这是一个非常简单的处理过程。}
    \label{fig:5-28}
\end{figure}
\begin{figure}[h]
    \includegraphics[width=\textwidth]{5-29}
    \centering
    \caption{齐次裁剪空间中$xw$平面上的截头体边界}
    \label{fig:5-29}
\end{figure}
[Blinn78]描述了如何在4D齐次空间中实现裁剪算法(图\ref{fig:5-29})。在透视除法之后,平截头体内的点$ (\frac{x}{w},\frac{y}{w},\frac{z}{w},1) $将位于规范化设备空间,它的边界如下:
\begin{align*}
-1\leq x/w \leq 1\\
-1\leq y/w \leq 1\\
0 \leq z/w \leq 1
\end{align*}
那么在透视除法之前,平截头体内的4D点(x , y , z , w)在齐次裁剪空间中的边界为:
\begin{align*}
-w\leq x \leq w\\
-w\leq y \leq w\\
0 \leq z \leq w
\end{align*}
也就是,顶点被限定在以下4D平面构成的空间范围内:
\begin{align*}
\mathit{Left:}&w=-x\\
\mathit{Right:}&w=x\\
\mathit{Bottom:}&w=-y\\
\mathit{Top:}&w=y\\
\mathit{Near:}&z=0\\
\mathit{Far:}&z=w
\end{align*}
只要我们知道齐次剪裁空间中的平截头体平面方程,我们就能使用任何一种裁剪算法(比如Sutherland-Hodgeman)。注意,由线段/平面相交测试的数学推论可知,这个测试在$\mathbf{R}^{4}$也能使用,所以我们可以在齐次裁剪空间中进行4D点和4D平面的相交测试。
\end{flushleft}

\section{光栅化阶段(The Rasterization Stage)}
\begin{flushleft}
光栅化(rasterization)阶段的主要任务是为投影后的3D三角形计算像素颜色。
\end{flushleft}
\subsection{视口变换(Viewport Transform)}
\begin{flushleft}
在裁剪之后,硬件会自动执行透视除法,将顶点从齐次裁剪空间变换到规范化设备空间(NDC)。一旦顶点进入NDC空间,构成2D图像的2D x、y坐标就会被变换到后台缓冲区中的一个称为视口的矩形区域内(回顾4.2.8节)。在该变换之后,$x$、$y$坐标将以像素为单位。通常,视口变换不修改$z$坐标,因为z坐标还要由深度缓存使用,但是我们可以通过 D3D12\_VIEWPORT 结构体的 MinDepth 和 MaxDepth 值修改z坐标的取值范围。MinDepth 和 MaxDepth 的值必须在0和1之间。
\end{flushleft}
\subsection{背面剔除(Backface Culling)}
\begin{flushleft}
一个三角形有两个面。我们使用如下约定来区分这两个面。假设三角形的顶点按照$v_{0}$、$v_{1}$、$v_{2}$的顺序排列,我们这样来计算三角形的法线$n$:
\begin{align*}
e_{0}&=v_{1}-v_{0}\\
e_{1}&=v_{2}-v_{1}\\
n&=\frac{e_{0}\times e_{1}}{||e_{0}\times e_{1}||}
\end{align*}
带有法线向量的面为正面,而另一个面为背面。图\ref{fig:5-30}说明了这一概念。
\begin{figure}[h]
    \includegraphics[width=\textwidth]{5-30}
    \centering
    \caption{左边的三角形正对我们的观察点,而右边的三角形背对我们的观察点。}
    \label{fig:5-30}
\end{figure}
当观察者看到三角形的正面时,我们说三角形是朝前的;当观察者看到三角形的背面时, 我们说三角形是朝后的。如图\ref{fig:5-30}所示,左边的三角形是朝前的,而右边的三角形是朝后的。而且,按照我们的观察角度,左边的三角形会按顺时针方向环绕,而右边的三角形会按逆时针方向环绕。这不是巧合:因为按照我们选择的约定(即,我们计算三角形法线的方式),按顺时针方向环绕的三角形(相对于观察者)是朝前的,而按逆时针方向环绕的三角形(相对于观察者)是朝后的。\\
~\\
现在,3D空间中的大部分物体都是封闭实心物体。当我们按照这一方式将每个三角形的法线指向物体外侧时,摄像机就不会看到实心物体朝后的三角形,因为朝前的三角形挡住了朝后的三角形;图\ref{fig:5-31}和图\ref{fig:5-32}分别以2D和3D形式说明了一概念。由于朝前的三角形挡住了朝后的三角形,所以绘制它们是毫无意义的。背面消隐(backface culling)是指让管线放弃对朝后的三角形的处理。这可以将所要处理的三角形的数量降低到原数量的一半。
\end{flushleft}
\begin{figure}[h]
    \includegraphics[width=\textwidth]{5-31}
    \centering
    \caption{(a)一个带有朝前和朝后三角形的实心物体。(b)在剔除了朝后的三角形之后的场景。注意,背面消隐不会影响最终的图像,因为朝后的三角形会被朝前的三角形阻挡。}
    \label{fig:5-31}
\end{figure}
\begin{figure}[h]
    \includegraphics[width=\textwidth]{5-32}
    \centering
    \caption{(左图)当以透明方式绘制立方体时,我们可以看到所有的6个面。(右图)当以实心方式绘制立方体时,我们无法看到朝后的3个面,因为朝前的3个面挡住了它们——所以朝后的三角形可以被直接丢弃,不再接受后续处理,没人能看到些朝后的三角形。}
    \label{fig:5-32}
\end{figure}
\begin{flushleft}
默认情况下,Direct3D将(相对于观察者)顺时针方向环绕的三角形视为朝前的三角形,将(相对于观察者)逆时针方向环绕的三角形视为朝后的三角形。不过,这一约定可以通过修改Direct3D渲染状态颠倒过来。
\end{flushleft}

\subsection{顶点属性插值(Vertex Attribute Interpolation)}
\begin{flushleft}
如前所述,我们通过指定三角形的3个顶点来定义一个三角形。除位置外,顶点还可以包含其他属性,比如颜色、法线向量和纹理坐标。在视口变换之后,这些属性必须为三角形表面上的每个像素进行插值。顶点深度值也必须进行插值,以使每个像素都有一个可用于深度缓存算法的深度值。对屏幕空间中的顶点属性进行插值,其实就是对3D空间中的三角形表面进行线性插值(如图\ref{fig:5-33}所示);这一工作需要借助所谓的透视矫正插值(perspective
correct interpolation)来实现。本质上,三角形表面内部的像素颜色都是通过顶点插值得到的。
\end{flushleft}
\begin{figure}[h]
    \includegraphics[width=\textwidth]{5-33}
    \centering
    \caption{通过对三角形顶点之间的属性值进行线性插值,可以得到三角形表面上的任一属性值p(s,t)。}
    \label{fig:5-33}
\end{figure}
\begin{flushleft}
我们不必关心透视精确插值的数学细节,因为硬件会自动完成这一工作;不过,有兴趣的读者可以在[Eberly01]中查阅相关的数学推导过程。图5.34介绍了一点基本思路:
\end{flushleft}
\begin{figure}[h]
    \includegraphics[width=\textwidth]{5-34}
    \centering
    \caption{一条3D线被投影到投影窗口上(在屏幕空间中投影是一条2D线)。我们看到,在3D线上取等距离的点,在2D屏幕空间上的投影点却不是等距离的。所以,我们在3D空间中执行线性插值,在屏幕空间需要执行非线性插值。}
    \label{fig:5-34}
\end{figure}

\section{像素着色器阶段(The Pixel Shader Stage)}
\begin{flushleft}
像素着色器(Pixel shader)是由我们编写的在GPU上执行的程序。像素着色器会处理每个像素片段(pixel fragment),它的输入是插值后的顶点属性,由此计算出一个颜色。像素着色器可以非常简单地输出一个颜色,也可以很复杂,例如实现逐像素光照、反射和阴影等效果。
\end{flushleft}
\section{输出合并阶段(The Output Merger Stage)}
\begin{flushleft}
当像素片段由像素着色器生成之后,它们会被传送到渲染管线的输出合并(output
merger,简称OM)阶段。在该阶段中,某些像素片段会被丢弃(例如,未能通过深度测试或模板测试)。未丢弃的像素片段会被写入后台缓冲区。混合(blending)工作是在该阶段中完成的,一个像素可以与后台缓冲区中的当前像素进行混合,并以混合后的值作为该像素的最终颜色。某些特殊效果,比如透明度,就是通过混合来实现的;我们会在第10章专门讲解混合。
\end{flushleft}