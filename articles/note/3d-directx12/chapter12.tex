\chapter{几何着色器(The Geometry Shader)}
\begin{flushleft}
假设我们没有使用曲面细分阶段,几何着色器阶段是位于顶点和像素着色器阶段之间的可选阶段。顶点着色器输入顶点时,几何着色器输入整个基本体(primitives)。 例如,如果我们绘制三角形列表,那么概念上将为列表中的每个三角形$T$执行几何着色器程序:\\
\end{flushleft}

\begin{lstlisting}
for (UINT i = 0; i < numTriangles; ++i)
  OutputPrimitiveList = GeometryShader(T[i].vertexList);
\end{lstlisting}

\begin{flushleft}
请注意,每个三角形的三个顶点都输入到几何着色器中,几何着色器输出一个基元列表(list of primitives)。 与不能破坏或创建顶点的顶点着色器不同,几何着色器的主要优点是它可以创建或破坏几何体; 这样就可以在GPU上实现一些有趣的效果。 例如,输入基元(primitive)可以扩展为一个或多个其他基元,或者几何着色器可以选择不基于某些条件输出基元。 注意,输出基元不需要与输入基元的类型相同; 例如,几何着色器的常见应用是将点扩展为四边形(两个三角形)。\\

从几何着色器输出的基元由顶点列表定义。 离开几何着色器的顶点位置必须转换为齐次剪辑空间。 在几何着色器阶段之后,我们有一个顶点列表,用于定义齐次剪辑空间中的基元。 投影这些顶点(齐次分割 homogeneous devide),然后像往常一样发生光栅化。
\end{flushleft}

{\large Objectives:}
\begin{itemize}
    \item 1.了解如何编程几何着色器。。
    \item 2.了解如何使用几何着色器有效地实现广告牌。
    \item 3.识别自动生成的基元ID及其某些应用程序。
    \item 4.了解如何创建和使用纹理数组,并了解它们的用途。
    \item 5.要了解alpha-to-coverage如何帮助解决alpha截断的锯齿问题。
\end{itemize}

%-------- 12.1 --------
\section{几何着色器编程(Programming Geometry Shaders)}
\begin{flushleft}
几何着色器编程很像顶点或像素着色器编程,但存在一些差异。以下代码显示了一般形式:\\
\end{flushleft}

\begin{lstlisting}
[maxvertexcount(N)]
void ShaderName (
    PrimitiveType InputVertexType InputName [NumElements],
    inout StreamOutputObject<OutputVertexType> OutputName)
{
    // Geometry shader body...
}
\end{lstlisting}

\begin{flushleft}
我们必须首先指定几何着色器将为单个调用输出的最大顶点数(每个基元调用几何着色器)。 这是通过使用以下属性语法在着色器定义之前设置最大顶点计数来完成的:\\
\end{flushleft}

\begin{lstlisting}
[maxvertexcount(N)]
\end{lstlisting}

\begin{flushleft}
其中$N$是几何着色器为单次调用输出的最大顶点数。几何着色器每次调用可以输出的顶点数是可变的,但不能超过定义的最大值。出于性能考虑,maxvertexcount 应尽可能小; [NVIDIA08]指出GS的峰值性能出现在GS输出1-20标量之间,如果GS输出在27-40个标量之间,性能会下降到50%。每次调用的标量输出数量是maxvertexcount和输出顶点类型结构中标量数量的乘积。在实践中使用这些限制很困难,因此我们可以接受低于峰值的性能,或者选择不使用几何着色器的替代实现;但是,我们还必须考虑替代实现可能具有其他缺点,这仍然可以使几何着色器实现成为更好的选择。此外,[NVIDIA08]中的建议来自2008年(第一代几何着色器),因此现在应该有所改进。\\

几何着色器采用两个参数:输入参数和输出参数。 (实际上,它可能需要更多,这是单独讨论的主题,参见§12.2.4。)输入参数始终是一个顶点数组,它定义一个基元——一个顶点,一个线为两个顶点,三个为a 三角形,四个用于邻接的线,六个用于相邻的三角形。 输入顶点的顶点类型是顶点着色器返回的顶点类型(例如,VertexOut)。 输入参数必须以基本类型为前缀,描述输入到几何着色器中的基元类型。 这可以是以下任何一种:\\
\end{flushleft}

\begin{itemize}
  \item 1.point: 输入基元是点。
  \item 2.line: 输入基元是线段(列表或条带)。
  \item 3.triangle: 输入基元是三角形(列表或条带)。
  \item 4.lineadj: 输入基元是邻接线段(列表或条带)。
  \item 5.triangleadj: 输入基元是邻接三角(列表或条带)。
\end{itemize}

\begin{flushleft}
~\\
NOTICE: 几何着色器中的输入基元始终是完整基元(例如,线的两个顶点和三角形的三个顶点)。 因此,几何着色器不需要区分列表和条带。 例如,如果要绘制三角形条带,则仍会对条带中的每个三角形执行几何着色器,并且每个三角形的三个顶点将作为输入传递到几何着色器中。 这需要额外的开销,因为多个基元共享的顶点在几何着色器中被多次处理。
~\\
输出参数始终具有inout修饰符。 此外,输出参数始终是流类型(stream type)。 流类型存储顶点列表,这些顶点定义几何着色器输出的几何。 几何着色器使用内在的Append方法将一个顶点添加到传出流列表:\\
\end{flushleft}

\begin{lstlisting}
void StreamOutputObject<OutputVertexType>::Append(OutputVertexType v);
\end{lstlisting}

\begin{flushleft}
流类型是模板类型,其中template参数用于指定传出顶点的顶点类型(例如,GeoOut)。 有三种可能的流类型:\\
\end{flushleft}

\begin{itemize}
  \item 1.PointStream<OutputVertexType>: 一个顶点列表定义的点列表(point list)
  \item 2.LineStream<OutputVertexType>: 一个顶点列表定义的线段条带(line strip)
  \item 3.TriangleStream<OutputVertexType>: 一个顶点列表定义的三角条带(triangle strip)
\end{itemize}

\begin{flushleft}
由几何着色器输出的顶点形成基元; 输出基元的类型由流类型(PointStream,LineStream,TriangleStream)指定。 对于线和三角形,输出基元始终是条带。 但是,可以使用内部RestartStrip方法模拟线和三角形列表:\\
\end{flushleft}

\begin{lstlisting}
void StreamOutputObject<OutputVertexType>::RestartStrip();
\end{lstlisting}

\begin{flushleft}
例:如果你想输出三角列表,那么你每次将三个顶点追加到输出流之后都需要调用 RestartStrip 方法。\\

下面是几何着色器签名的例子:\\
\end{flushleft}

\begin{lstlisting}
// EXAMPLE 1: GS ouputs at most 4 vertices. The input
// primitive is a line.
// The output is a triangle strip.
//
[maxvertexcount(4)]
void GS(line VertexOut gin[2],
    inout TriangleStream<GeoOut> triStream)
{
    // Geometry shader body…
}
//
// EXAMPLE 2: GS outputs at most 32 vertices. The
// input primitive is a triangle. The output 
// is a triangle strip.
//
[maxvertexcount(32)]
void GS(triangle VertexOut gin[3],
    inout TriangleStream<GeoOut> triStream)
{
    // Geometry shader body…
}
//
// EXAMPLE 3: GS outputs at most 4 vertices. The input
// primitive is a point. The output is a triangle strip.
//
[maxvertexcount(4)]
void GS(point VertexOut gin[1],
    inout TriangleStream<GeoOut> triStream)
{
    // Geometry shader body…
}
\end{lstlisting}

\begin{figure}[h]
    \includegraphics[width=\textwidth]{12-1}
    \centering
    \caption{将三角形细分为四个大小相等的三角形。 观察到三个新顶点是沿着原始三角形边缘的中点。}
    \label{fig:12-1}
\end{figure}

\begin{flushleft}
以下几何着色器说明了Append和RestartStrip方法; 它输入一个三角形,将其细分(图\ref{fig:12-1})并输出四个细分的三角形:\\
\end{flushleft}

\begin{lstlisting}
struct VertexOut
{
    float3 PosL : POSITION;
    float3 NormalL : NORMAL;
    float2 Tex : TEXCOORD;
};
struct GeoOut
{
    float4 PosH : SV_POSITION;
    float3 PosW : POSITION;
    float3 NormalW : NORMAL;
    float2 Tex : TEXCOORD;
    float FogLerp : FOG;
};
void Subdivide(
    VertexOut inVerts[3], 
    out VertexOut
    outVerts[6])
{
    //       1
    //       *
    //     /   \
    //    /     \
    //   m0*–—--*m1
    //  /   \   / \
    // /     \ /   \
    // *–—*–—*
    // 0     m2     2
    VertexOut m[3];
    // Compute edge midpoints.
    m[0].PosL = 0.5f*(inVerts[0].PosL+inVerts[1].PosL);
    m[1].PosL = 0.5f*(inVerts[1].PosL+inVerts[2].PosL);
    m[2].PosL = 0.5f*(inVerts[2].PosL+inVerts[0].PosL);
    // Project onto unit sphere
    m[0].PosL = normalize(m[0].PosL);
    m[1].PosL = normalize(m[1].PosL);
    m[2].PosL = normalize(m[2].PosL);
    // Derive normals.
    m[0].NormalL = m[0].PosL;
    m[1].NormalL = m[1].PosL;
    m[2].NormalL = m[2].PosL;
     // Interpolate texture coordinates.
    m[0].Tex = 0.5f*(inVerts[0].Tex+inVerts[1].Tex);
    m[1].Tex = 0.5f*(inVerts[1].Tex+inVerts[2].Tex);
    m[2].Tex = 0.5f*(inVerts[2].Tex+inVerts[0].Tex);
    outVerts[0] = inVerts[0];
    outVerts[1] = m[0];
    outVerts[2] = m[2];
    outVerts[3] = m[1];
    outVerts[4] = inVerts[2];
    outVerts[5] = inVerts[1];
};
void OutputSubdivision(VertexOut v[6],
    inout TriangleStream<GeoOut> triStream)
{
    GeoOut gout[6];
    [unroll]
    for(int i = 0; i < 6; ++i)
    {
        // Transform to world space space.
        gout[i].PosW = mul(float4(v[i].PosL, 1.0f),
        gWorld).xyz;
        gout[i].NormalW = mul(v[i].NormalL, (float3x3)gWorldInvTranspose);
        // Transform to homogeneous clip space.
        gout[i].PosH = mul(float4(v[i].PosL, 1.0f),
        gWorldViewProj);
        gout[i].Tex = v[i].Tex;
    }
    //       1
    //       *
    //     /   \
    //    /     \
    //   m0*–—--*m1
    //  /   \   / \
    // /     \ /   \
    // *–—*–—*
    // 0     m2     2
    // We can draw the subdivision in two strips:
    // Strip 1: bottom three triangles
    // Strip 2: top triangle
    [unroll]
    for(int j = 0; j < 5; ++j)
    {
        triStream.Append(gout[j]);
    }
    triStream.RestartStrip();
    triStream.Append(gout[1]);
    triStream.Append(gout[5]);
    triStream.Append(gout[3]);
}
[maxvertexcount(8)]
void GS(triangle VertexOut gin[3], inout
    TriangleStream<GeoOut>)
{
    VertexOut v[6];
    Subdivide(gin, v);
    OutputSubdivision(v, triStream);
}
\end{lstlisting}

\begin{flushleft}
几何着色器的编译方式与顶点和像素着色器非常相似。 假设我们在TreeSprite.hlsl中有一个名为GS的几何着色器,那么我们将着色器编译为字节码,如下所示:\\
\end{flushleft}

\begin{lstlisting}
mShaders["treeSpriteGS"] = d3dUtil::CompileShader(
    L"Shaders\\TreeSprite.hlsl", nullptr, "GS", "gs_5_0");
\end{lstlisting}

\begin{flushleft}
与顶点和像素着色器一样,给定的几何着色器作为管道状态对象(PSO)的一部分绑定到渲染管道:\\
\end{flushleft}

\begin{lstlisting}
D3D12_GRAPHICS_PIPELINE_STATE_DESC treeSpritePsoDesc =
    opaquePsoDesc;
...
treeSpritePsoDesc.GS =
{
    reinterpret_cast<BYTE*>(mShaders["treeSpriteGS"]->
        GetBufferPointer()),
    mShaders["treeSpriteGS"]->GetBufferSize()
};
\end{lstlisting}

\begin{flushleft}
~\\
NOTICE: 给定输入基元,几何着色器可以选择不根据某些条件输出它。 通过这种方式,几何体被几何着色器"破坏",这对某些算法很有用。\\
~\\

NOTICE: 如果没有输出足够的顶点来完成几何着色器中的基元,则会丢弃部分基元。\\
\end{flushleft}

%-------- 12.2 --------
\section{树广告牌演示(Tree Billboards demo)}
%-------- 12.2.1 --------
\subsection{概览(Overview)}
\begin{flushleft}
当树木很远时,使用广告牌技术来提高效率。 也就是说,不是渲染完全3D树的几何图形,而是在其上绘制带有3D树图片的四边形(参见图\ref{fig:12-2})。 从远处看,您无法分辨出正在使用的广告牌。 然而,诀窍是确保广告牌始终面向相机(否则达不到欺骗人眼的目的)。
\end{flushleft}

\begin{figure}[h]
    \includegraphics[width=\textwidth]{12-2}
    \centering
    \caption{一个alpha通道的树广告牌纹理}
    \label{fig:12-2}
\end{figure}

\begin{flushleft}
假设$y$轴向上并且$xz$平面是地平面,则树广告牌通常将与$y$轴对齐并且恰好面向$xz$平面中的相机。 图\ref{fig:12-3}显示了鸟瞰图中几个广告牌的局部坐标系,请注意广告牌正在"看"相机。\\
\end{flushleft}

\begin{figure}[h]
    \includegraphics[width=\textwidth]{12-3}
    \centering
    \caption{树广告牌朝向相机}
    \label{fig:12-3}
\end{figure}

\begin{flushleft}
因此,假设世界空间中广告牌的中心位置$C=(C_{x},C_{y},C_{z})$和世界空间中相机$E=(E_{x},E_{y},E_{z})$的位置,我们有足够的信息描述广告牌相对于世界空间的局部坐标系:\\
\end{flushleft}

\begin{align*}
W&=\frac{(E_{x}-C_{x},0,E_{z}-C_{z})}{||(E_{x}-C_{x},0,E_{z}-C_{z})||}\\
v&=(0,1,0)\\
u&=v\times w
\end{align*}

\begin{flushleft}
给定广告牌相对于世界空间的局部坐标系,以及广告牌的世界尺寸,广告牌四边形顶点可以如下获得(见图\ref{fig:12-4}):\\
\end{flushleft}

\begin{lstlisting}
v[0] = float4(gin[0].CenterW + halfWidth*right - halfHeight*up, 1.0f);
v[1] = float4(gin[0].CenterW + halfWidth*right + halfHeight*up, 1.0f);
v[2] = float4(gin[0].CenterW - halfWidth*right - halfHeight*up, 1.0f);
v[3] = float4(gin[0].CenterW - halfWidth*right + halfHeight*up, 1.0f);
\end{lstlisting}

\begin{figure}[h]
    \includegraphics[width=\textwidth]{12-4}
    \centering
    \caption{从本地坐标系统和广告牌的世界大小计算广告牌四边形顶点。}
    \label{fig:12-4}
\end{figure}

\begin{figure}[h]
    \includegraphics[width=\textwidth]{12-5}
    \centering
    \caption{树广告牌演示截图}
    \label{fig:12-5}
\end{figure}

\begin{flushleft}
请注意,广告牌的局部坐标系对于每个广告牌都不同,因此必须为每个广告牌计算。\\

对于本演示,我们将构建一个点基元列表(PSO的PrimitiveTopologyType的D3D12\_PRIMITIVE\_TOPOLOGY\_TYPE\_POINT和作为ID3D12GraphicsCommandList::IASetPrimitiveTopology的参数的D3D\_PRIMITIVE\_TOPOLOGY\_POINTLIST)略高于陆地平面(land mass)。 这些点代表我们想要绘制的广告牌的中心。 在几何着色器中,我们将这些点扩展为广告牌四边形。 此外,我们将在几何着色器中计算广告牌的世界矩阵。 图\ref{fig:12-5}显示了演示的屏幕截图。\\

如图\ref{fig:12-5}所示,此示例构建了第10章的“Blend”演示。\\

~\\
NOTICE: 广告牌的常见CPU实现是在动态顶点缓冲区(即上载堆)中每个广告牌使用四个顶点。 然后每次相机移动时,顶点都会在CPU上更新并存储到GPU缓冲区,以便广告牌面向相机。 此方法必须将每个广告牌的四个顶点提交到IA阶段,并且需要更新具有开销的动态顶点缓冲区。 使用几何着色器方法,可以利用静态顶点缓冲区,几何着色器执行广告牌扩展并使广告牌面向相机。 此外,广告牌的内存占用量非常小,只需要将每个广告牌的一个顶点提交到IA阶段即可。
~\\
\end{flushleft}

%-------- 12.2.2 --------
\subsection{顶点结构(Vertex Structure)}
\begin{lstlisting}
struct TreeSpriteVertex
{
    XMFLOAT3 Pos;
    XMFLOAT2 Size;
};

mTreeSpriteInputLayout =
{
    {"POSITION", 0, DXGI_FORMAT_R32G32B32_FLOAT, 0, 0,
                 D3D12_INPUT_CLASSIFICATION_PER_VERTEX_DATA, 0},
    {"SIZE",     0, DXGI_FORMAT_R32G32_FLOAT, 0, 12,
                 D3D12_INPUT_CLASSIFICATION_PER_VERTEX_DATA, 0}
};
\end{lstlisting}

\begin{flushleft}
顶点(vertex)存储一个点(point),该点表示广告牌在世界空间中的中心位置。它还包括一个尺寸分量,它存储广告牌的宽度/高度(按比例缩放到世界空间单位);也就是几何着色器知道扩展后广告牌应该有多大(图\ref{fig:12-6})。通过使每个顶点的尺寸不同,我们可以轻松地容纳不同尺寸的广告牌。\\
\end{flushleft}

\begin{figure}[h]
    \includegraphics[width=\textwidth]{12-6}
    \centering
    \caption{将一个点扩展为四边形。}
    \label{fig:12-6}
\end{figure}

\begin{flushleft}
除了纹理数组(第12.2.4节)之外,"Tree Billboard"演示中的其他C++代码现在应该是常规的Direct3D代码(创建顶点缓冲区,效果,调用绘制方法等)。因此,我们现在将注意力转向TreeSprite.hlsl文件。
\end{flushleft}

%-------- 12.2.3 --------
\subsection{对应的HLSL文件(The HLSL File)}
\begin{flushleft}
由于这是我们的第一个带有几何着色器的演示,我们将在此处显示整个HLSL文件,以便您可以看到它如何与顶点和像素着色器组合在一起。这个效果还引入了一些我们尚未讨论过的新对象(SV_PrimitiveID和Texture2DArray);这些项目将在下面讨论。目前,主要关注几何着色器程序GS;此着色器将一个点扩展为与面向摄像机的世界y轴对齐的四边形,如12.2.1中所述。\\
\end{flushleft}

\begin{lstlisting}
//************************************************************
// TreeSprite.hlsl by Frank Luna (C) 2015 All Rights Reserved.
//************************************************************

// Defaults for number of lights.
#ifndef NUM_DIR_LIGHTS
    #define NUM_DIR_LIGHTS 3
#endif

#ifndef NUM_POINT_LIGHTS
    #define NUM_POINT_LIGHTS 0
#endif

#ifndef NUM_SPOT_LIGHTS
    #define NUM_SPOT_LIGHTS 0
#endif

// Include structures and functions for lighting.
#include "LightingUtil.hlsl"

Texture2DArray gTreeMapArray : register(t0);


SamplerState gsamPointWrap        : register(s0);
SamplerState gsamPointClamp       : register(s1);
SamplerState gsamLinearWrap       : register(s2);
SamplerState gsamLinearClamp      : register(s3);
SamplerState gsamAnisotropicWrap  : register(s4);
SamplerState gsamAnisotropicClamp : register(s5);

// Constant data that varies per frame.
cbuffer cbPerObject : register(b0)
{
    float4x4 gWorld;
    float4x4 gTexTransform;
};

// Constant data that varies per material.
cbuffer cbPass : register(b1)
{
    float4x4 gView;
    float4x4 gInvView;
    float4x4 gProj;
    float4x4 gInvProj;
    float4x4 gViewProj;
    float4x4 gInvViewProj;
    float3 gEyePosW;
    float cbPerObjectPad1;
    float2 gRenderTargetSize;
    float2 gInvRenderTargetSize;
    float gNearZ;
    float gFarZ;
    float gTotalTime;
    float gDeltaTime;
    float4 gAmbientLight;

    float4 gFogColor;
    float gFogStart;
    float gFogRange;
    float2 cbPerObjectPad2;

    // Indices [0, NUM_DIR_LIGHTS) are directional lights;
    // indices [NUM_DIR_LIGHTS, NUM_DIR_LIGHTS+NUM_POINT_LIGHTS) are point lights;
    // indices [NUM_DIR_LIGHTS+NUM_POINT_LIGHTS, NUM_DIR_LIGHTS+NUM_POINT_LIGHT+NUM_SPOT_LIGHTS)
    // are spot lights for a maximum of MaxLights per object.
    Light gLights[MaxLights];
};

cbuffer cbMaterial : register(b2)
{
    float4   gDiffuseAlbedo;
    float3   gFresnelR0;
    float    gRoughness;
    float4x4 gMatTransform;
};
 
struct VertexIn
{
    float3 PosW  : POSITION;
    float2 SizeW : SIZE;
};

struct VertexOut
{
    float3 CenterW : POSITION;
    float2 SizeW   : SIZE;
};

struct GeoOut
{
    float4 PosH    : SV_POSITION;
    float3 PosW    : POSITION;
    float3 NormalW : NORMAL;
    float2 TexC    : TEXCOORD;
    uint   PrimID  : SV_PrimitiveID;
};

VertexOut VS(VertexIn vin)
{
    VertexOut vout;

    // Just pass data over to geometry shader.
    vout.CenterW = vin.PosW;
    vout.SizeW   = vin.SizeW;

    return vout;
}
 
 // We expand each point into a quad (4 vertices), so the maximum number of vertices
 // we output per geometry shader invocation is 4.
[maxvertexcount(4)]
void GS(point VertexOut gin[1], 
        uint primID : SV_PrimitiveID, 
        inout TriangleStream<GeoOut> triStream)
{   
    //
    // Compute the local coordinate system of the sprite relative to the world
    // space such that the billboard is aligned with the y-axis and faces the eye.
    //

    float3 up = float3(0.0f, 1.0f, 0.0f);
    float3 look = gEyePosW - gin[0].CenterW;
    look.y = 0.0f; // y-axis aligned, so project to xz-plane
    look = normalize(look);
    float3 right = cross(up, look);

    //
    // Compute triangle strip vertices (quad) in world space.
    //
    float halfWidth  = 0.5f*gin[0].SizeW.x;
    float halfHeight = 0.5f*gin[0].SizeW.y;
    
    float4 v[4];
    v[0] = float4(gin[0].CenterW + halfWidth*right - halfHeight*up, 1.0f);
    v[1] = float4(gin[0].CenterW + halfWidth*right + halfHeight*up, 1.0f);
    v[2] = float4(gin[0].CenterW - halfWidth*right - halfHeight*up, 1.0f);
    v[3] = float4(gin[0].CenterW - halfWidth*right + halfHeight*up, 1.0f);

    //
    // Transform quad vertices to world space and output 
    // them as a triangle strip.
    //
    
    float2 texC[4] = 
    {
        float2(0.0f, 1.0f),
        float2(0.0f, 0.0f),
        float2(1.0f, 1.0f),
        float2(1.0f, 0.0f)
    };
    
    GeoOut gout;
    [unroll]
    for(int i = 0; i < 4; ++i)
    {
        gout.PosH     = mul(v[i], gViewProj);
        gout.PosW     = v[i].xyz;
        gout.NormalW  = look;
        gout.TexC     = texC[i];
        gout.PrimID   = primID;
        
        triStream.Append(gout);
    }
}

float4 PS(GeoOut pin) : SV_Target
{
    float3 uvw = float3(pin.TexC, pin.PrimID%3);
    float4 diffuseAlbedo = gTreeMapArray.Sample(gsamAnisotropicWrap, uvw) * gDiffuseAlbedo;

#ifdef ALPHA_TEST
    // Discard pixel if texture alpha < 0.1.  We do this test as soon 
    // as possible in the shader so that we can potentially exit the
    // shader early, thereby skipping the rest of the shader code.
    clip(diffuseAlbedo.a - 0.1f);
#endif

    // Interpolating normal can unnormalize it, so renormalize it.
    pin.NormalW = normalize(pin.NormalW);

    // Vector from point being lit to eye. 
    float3 toEyeW = gEyePosW - pin.PosW;
    float distToEye = length(toEyeW);
    toEyeW /= distToEye; // normalize

    // Light terms.
    float4 ambient = gAmbientLight*diffuseAlbedo;

    const float shininess = 1.0f - gRoughness;
    Material mat = { diffuseAlbedo, gFresnelR0, shininess };
    float3 shadowFactor = 1.0f;
    float4 directLight = ComputeLighting(gLights, mat, pin.PosW,
        pin.NormalW, toEyeW, shadowFactor);

    float4 litColor = ambient + directLight;

#ifdef FOG
    float fogAmount = saturate((distToEye - gFogStart) / gFogRange);
    litColor = lerp(litColor, gFogColor, fogAmount);
#endif

    // Common convention to take alpha from diffuse albedo.
    litColor.a = diffuseAlbedo.a;

    return litColor;
}
\end{lstlisting}

%-------- 12.2.4 --------
\subsection{SV_PrimitiveID}
\begin{flushleft}
此示例中的几何着色器采用带有语义 SV_PrimitiveID 的特殊无符号整数参数。
\end{flushleft}

\begin{lstlisting}[escapechar=^]
[maxvertexcount(4)]
void GS(point VertexOut gin[1],
    ^\textbf{uint primID : SV_PrimitiveID,}^
    inout TriangleStream<GeoOut> triStream)
\end{lstlisting}

\begin{flushleft}
指定此语义时,它会告诉输入汇编程序阶段为每个基元自动生成基元 ID。当执行绘制调用以绘制$n$个基元时,第一个基元标记为$0$;第二个基元标记为$1$;依此类推,直到绘制调用中的最后一个原语标记为$n-1$。原始ID仅对于单个绘制调用是唯一的。在我们的广告牌示例中,几何着色器不使用此ID(尽管几何着色器可以);相反,几何着色器将基元ID写入传出顶点,从而将其传递到像素着色器阶段。像素着色器使用基元ID索引到纹理数组,这在下一章节说明。\\

~\\
NOTICE: 如果不存在几何着色器,则可以将原始ID参数添加到像素着色器的参数列表中:\\
\end{flushleft}

\begin{lstlisting}
float4 PS(VertexOut pin, 
    uint primID : SV_PrimitiveID) : SV_Target
{
    // Pixel shader body...
}
\end{lstlisting}

\begin{flushleft}
但是,如果存在几何着色器,则基元ID参数必须出现在几何着色器签名中。然后,几何着色器可以使用基元ID或将其传递给像素着色器阶段(或两者)。\\

输入汇编程序也可以生成顶点ID。为此,请使用语义SV\_VertexID 向顶点着色器签名添加类型为 uint 的参数:\\
\end{flushleft}

\begin{lstlisting}
VertexOut VS(VertexIn vin, uint vertID : SV_VertexID)
{
    // vertex shader body...
}
\end{lstlisting}

\begin{flushleft}
对于Draw调用,draw 调用中的顶点将使用$0,1,...,n-1$中的ID进行标记,其中$n$是绘制调用中的顶点数。对于DrawIndexed调用,顶点ID对应于顶点索引值。\\
\end{flushleft}

%-------- 12.3 --------
\section{纹理数组(Texture Arrays)}




