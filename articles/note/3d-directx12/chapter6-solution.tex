\chapter*{6.13 习题}
%---------- 1 ----------------
\begin{flushleft}
1. 写出下面顶点结构的 D3D12\_INPUT\_ELEMENT\_DESC 数组:
\end{flushleft}
\begin{lstlisting}
struct Vertex
{
    XMFLOAT3 Pos;
    XMFLOAT3 Tangent;
    XMFLOAT3 Normal;
    XMFLOAT2 Tex0;
    XMFLOAT2 Tex1;
    XMCOLOR  Color;
};
\end{lstlisting}
\begin{flushleft}
答:\\
\end{flushleft}
\begin{lstlisting}
std::vector<D3D12_INPUT_ELEMENT_DESC> mInputLayout;
mInputLayout = {
    { "POSITION", 0, DXGI_FORMAT_R32G32B32_FLOAT, 0, 0, D3D12_INPUT_CLASSIFICATION_PER_VERTEX_DATA, 0 },
    { "TANGENT", 0, DXGI_FORMAT_R32G32B32_FLOAT, 0, 12, D3D12_INPUT_CLASSIFICATION_PER_VERTEX_DATA, 0 },
    { "NORMAL", 0, DXGI_FORMAT_R32G32B32_FLOAT, 0, 24, D3D12_INPUT_CLASSIFICATION_PER_VERTEX_DATA, 0 },
    { "TEXCOORD", 0, DXGI_FORMAT_R32G32_FLOAT, 0, 36, D3D12_INPUT_CLASSIFICATION_PER_VERTEX_DATA, 0 },
    { "TEXCOORD", 1, DXGI_FORMAT_R32G32_FLOAT, 0, 44, D3D12_INPUT_CLASSIFICATION_PER_VERTEX_DATA, 0 },
    { "COLOR", 0, DXGI_FORMAT_R8G8B8A8_UINT, 0, 52, D3D12_INPUT_CLASSIFICATION_PER_VERTEX_DATA, 0 }
};
\end{lstlisting}

%---------- 2 ----------------
\begin{flushleft}
~\\
2. 重写彩箱DEMO,这次使用 2 个顶点缓冲区(2个输入槽)给管道提供顶点数据。一个缓冲区存储位置元素,另一个存储存储颜色元素。下面给定两个分开的顶点数据结构:\\
\end{flushleft}
\begin{lstlisting}
struct VPosData
{
    XMFLOAT3 Pos;
};

struct VColorData
{
    XMFLOAT4 Color;
};
\end{lstlisting}
\begin{flushleft}
位置元素挂在输入槽0上,颜色元素挂在输入操1上。此外还需注意 D3D12\_INPUT\_ELEMENT\_DESC::AlignedByteOffset 对于两个元素来说都是0;然后使用 ID3D12CommandList::IASetVertexBuffers 方法来将缓冲区绑定到槽0和槽1。然后,Direct3D将使用来自不同输入槽的元素来组合顶点。 这可以用作优化。 例如,在阴影映射算法中,我们需要每帧绘制两次场景:一次从光源的角度(阴影传递),一次从主摄像头的角度(主传递)。 阴影传递仅需要位置数据和纹理坐标(对于经过alpha测试的几何体)。 因此我们可以将顶点数据分成两个槽:一个槽包含位置和纹理坐标,另一个槽包含其他顶点属性(例如,法线和切线矢量)。 现在我们可以轻松地仅在阴影传递所需的顶点数据中流动(位置和纹理坐标),从而为阴影传递节省数据带宽。 主渲染过程将使用两个顶点输入槽来获取所需的所有顶点数据。 为了提高性能,建议最小化用于小于或等于3的小数字的输入槽数。\\
\end{flushleft}
\begin{flushleft}
答:TODO
\end{flushleft}

%---------- 3 ----------------
\begin{flushleft}
3. 绘制以下图形:\\
\begin{itemize}
  \item 1. 一个点列表,如图5.13a。
  \item 2. 一个线段条,如图5.13b。
  \item 3. 一个线段列表,如图5.13c。
  \item 4. 一个三角条,如图5.13d。
  \item 5. 一个三角列表,如图5.14a。
\end{itemize}
答: TODO
\end{flushleft}

%---------- 4 ----------------
\begin{flushleft}
4. 构造金字塔的顶点和索引列表,如图\ref{fig:6-8}所示,并绘制它。 将基顶点设为绿色和顶上顶点设为红色。\\
答:TODO
\end{flushleft}
\begin{figure}[h]
    \includegraphics[width=\textwidth]{6-8}
    \centering
    \caption{金字塔三角形}
    \label{fig:6-8}
\end{figure}
%---------- 5 ----------------
\begin{flushleft}
5. 运行“Box” DEMO,并回想一下我们仅在顶点指定颜色。 解释如何为三角形上的每个像素获取像素颜色。
\end{flushleft}


%---------- 6 ----------------
\begin{flushleft}
6. 修改 Box Demo, 在转换为世界空间之前将以下变换应用于顶点着色器中的每个顶点。
\end{flushleft}
\begin{lstlisting}
vin.PosL.xy += 0.5f*sin(vinL.Pos.x)*sin(3.0f*gTime);
vin.PosL.z *= 0.6f + 0.4f*sin(2.0f*gTime);
\end{lstlisting}
\begin{flushleft}
您需要添加一个gTime常量缓冲区变量; 此变量对应于当前的 GameTimer::TotalTime() 值。 使用正弦函数周期性地扭曲顶点来将时间通过顶点动画来展现。
\end{flushleft}

%---------- 7 ----------------
\begin{flushleft}
7. 将箱(box)和金字塔的顶点(练习4)合并到一个大的顶点缓冲区中。 还将框和金字塔的索引合并到一个大索引缓冲区中(但不更新索引值)。 然后使用 ID3D12CommandList::DrawIndexedInstanced 的参数逐个绘制框和金字塔。 使用世界变换矩阵,使框和金字塔在世界空间中不相交。
\end{flushleft}

%---------- 8 ----------------
\begin{flushleft}
8. 通过在线框(wireframe)模式下渲染多维数据集来修改Box演示。
\end{flushleft}

%---------- 9 ----------------
\begin{flushleft}
9. 修改Box演示, 禁用背面剔除(D3D12\_CULL\_NONE); 也尝试剔除正面而不是背面(D3D12\_CULL\_FRONT)。 以线框(wireframe)模式输出结果,以便您可以更轻松地查看差异。
\end{flushleft}

%---------- 10 ----------------
\begin{flushleft}
10. 如果顶点内存很重要,那么从128位颜色值减少到32位颜色值可能是值得的。修改“Box”演示 在顶点结构中使用32位颜色值而不是128位颜色值。 您的顶点结构和相应的顶点输入描述将如下所示:\\
\end{flushleft}
\begin{lstlisting}
struct Vertex
{
    XMFLOAT3 Pos;
    XMCOLOR Color;
}

D3D12_INPUT_ELEMENT_DESC vertexDesc[] = {
    {“POSITION”, 0, DXGI_FORMAT_R32G32B32_FLOAT, 0, 0, 
                 D3D12_INPUT_PER_VERTEX_DATA, 0},
    {“COLOR”,    0, DXGI_FORMAT_B8G8R8A8_UNORM, 0, 12,
                 D3D12_INPUT_PER_VERTEX_DATA, 0}
};
\end{lstlisting}
\begin{flushleft}
我们使用 DXGI\_FORMAT\_B8G8R8A8\_UNORM 格式(8位红色,绿色,蓝色和alpha)。 此格式对应于常见的32位图形颜色格式ARGB,但 DXGI\_FORMAT 符号以小端表示法列出它们在内存中出现的字节。 在little-endian中,多字节(multi-byte)数据字(word)的字节从最低有效字节写入最高有效字节,这就是为什么 ARGB 在内存中出现为 BGRA,其中最小内存地址处的最低有效字节和最高有效字节为 最高的内存地址。
\end{flushleft}

%---------- 11 ----------------
\begin{flushleft}
11. 思考下面 C++ 顶点结构:
\end{flushleft}
\begin{lstlisting}
struct Vertex
{
    XMFLOAT3 Pos;
    XMFLOAT4 Color;
};
\end{lstlisting}
\begin{itemize}
  \item 1. 输入布局描述顺序是否需要匹配顶点结构顺序? 也就是说,以下顶点声明是否适用于此顶点结构? 做一个实验来找出答案。 然后给出你为什么认为它有效或无效的推理。
  \begin{lstlisting}
  D3D11_INPUT_ELEMENT_DESC vertexDesc[] =
  {
      {“COLOR”,    0, DXGI_FORMAT_R32G32B32A32_FLOAT, 0, 12,
                   D3D11_INPUT_PER_VERTEX_DATA, 0},
      {“POSITION”, 0, DXGI_FORMAT_R32G32B32_FLOAT, 0, 0,
                   D3D11_INPUT_PER_VERTEX_DATA, 0}
  };
  \end{lstlisting}
  \item 2. 相应的顶点着色器结构顺序是否需要匹配 C++ 顶点结构顺序? 也就是说,以下顶点着色器结构是否与上述 C++ 顶点结构一起使用? 做一个实验来找出答案。 然后给出你为什么认为它有效或无效的推理。
  \begin{lstlisting}
  struct VertexIn
  {
      float4 Color : COLOR;
      float3 Pos   : POSITION;
  };
  \end{lstlisting}
\end{itemize}

%---------- 12 ----------------
\begin{flushleft}
12. 将视口(viewport)设置为后缓冲区(back buffer)的左半部分。
\end{flushleft}

%---------- 13 ----------------
\begin{flushleft}
13. 使用剪刀测试来剔除以后缓冲区为中心的矩形外的所有像素,宽度为mClientWidth/2,高度为 mClientHeight/2。 请记住,您还需要使用光栅化器状态组启用剪刀测试。
\end{flushleft}

%---------- 14 ----------------
\begin{flushleft}
14. 像素着色器颜色色调。 使用常量缓冲区为颜色随时间变化。 使用平滑缓动功能。 在顶点着色器和像素着色器中执行此操作。
\end{flushleft}

%---------- 15 ----------------
\begin{flushleft}
15. 修改 Box DEMO 中的像素着色器为如下形式:
\end{flushleft}
\begin{lstlisting}
float4 PS(VertexOut pin) : SV_Target
{
    clip(pin.Color.r - 0.5f);
    return pin.Color;
}
\end{lstlisting}
\begin{flushleft}
运行程序并猜测内置 clip 方法的作用。
\end{flushleft}

%---------- 16 ----------------
\begin{flushleft}
修改 Box 演示中的像素着色器,以在插值顶点颜色和通过常量缓冲区指定的 gPulseColor 之间平滑脉冲。 您还需要更新应用程序端的常量缓冲区。 HLSL代码中的常量缓冲区和像素着色器应如下所示:
\end{flushleft}
\begin{lstlisting}
cbuffer cbPerObject : register(b0)
{
    float4x4 gWorldViewProj;
    float4 gPulseColor;
    float gTime;
};

float4 PS(VertexOut pin) : SV_Target
{
    const float pi = 3.14159;

    // Oscillate a value in [0,1] over time using a sine functio.
    float s = 0.5f*sin(2*gTime - 0.25f(pi) + 0.5f;

    // Linearly interpolate between pin.Color and gPulseColor based on
    // parameter s.
    float4 c = lerp(pin.Color, gPulseColor, s);
    return c;
}
\end{lstlisting}
\begin{flushleft}
gTime 变量对应于 GameTimer::TotalTime() 的值。
\end{flushleft}
