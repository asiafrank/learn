\chapter{6.13 习题}
%---------- 1 ----------------
\begin{flushleft}
1. 写出下面顶点结构的 D3D12\_INPUT\_ELEMENT\_DESC 数组:
\end{flushleft}
\begin{lstlisting}
struct Vertex
{
    XMFLOAT3 Pos;
    XMFLOAT3 Tangent;
    XMFLOAT3 Normal;
    XMFLOAT2 Tex0;
    XMFLOAT2 Tex1;
    XMCOLOR  Color;
};
\end{lstlisting}
\begin{flushleft}
解:\\
\end{flushleft}
\begin{lstlisting}
std::vector<D3D12_INPUT_ELEMENT_DESC> mInputLayout;
mInputLayout = {
    { "POSITION", 0, DXGI_FORMAT_R32G32B32_FLOAT, 0, 0, 
                  D3D12_INPUT_CLASSIFICATION_PER_VERTEX_DATA, 0 },
    { "TANGENT",  0, DXGI_FORMAT_R32G32B32_FLOAT, 0, 12, 
                  D3D12_INPUT_CLASSIFICATION_PER_VERTEX_DATA, 0 },
    { "NORMAL",   0, DXGI_FORMAT_R32G32B32_FLOAT, 0, 24, 
                  D3D12_INPUT_CLASSIFICATION_PER_VERTEX_DATA, 0 },
    { "TEXCOORD", 0, DXGI_FORMAT_R32G32_FLOAT, 0, 36, 
                  D3D12_INPUT_CLASSIFICATION_PER_VERTEX_DATA, 0 },
    { "TEXCOORD", 1, DXGI_FORMAT_R32G32_FLOAT, 0, 44, 
                  D3D12_INPUT_CLASSIFICATION_PER_VERTEX_DATA, 0 },
    { "COLOR",    0, DXGI_FORMAT_R8G8B8A8_UINT, 0, 52, 
                  D3D12_INPUT_CLASSIFICATION_PER_VERTEX_DATA, 0 }
};
\end{lstlisting}

%---------- 2 ----------------
\begin{flushleft}
~\\
2. 重写Box DEMO,这次使用 2 个顶点缓冲区(2个输入槽)给管道提供顶点数据。一个缓冲区存储位置元素,另一个存储存储颜色元素。下面给定两个分开的顶点数据结构:\\
\end{flushleft}
\begin{lstlisting}
struct VPosData
{
    XMFLOAT3 Pos;
};

struct VColorData
{
    XMFLOAT4 Color;
};
\end{lstlisting}
\begin{flushleft}
位置元素挂在输入槽0上,颜色元素挂在输入槽1上。此外还需注意 D3D12\_INPUT\_ELEMENT\_DESC::AlignedByteOffset 对于两个元素来说都是0;然后使用 ID3D12CommandList::IASetVertexBuffers 方法来将缓冲区绑定到槽0和槽1。然后,Direct3D将使用来自不同输入槽的元素来组合顶点。 这可以用作优化。 例如,在阴影映射算法中,我们需要每帧绘制两次场景:一次从光源的角度(阴影传递),一次从主摄像头的角度(主传递)。 阴影传递仅需要位置数据和纹理坐标(对于经过alpha测试的几何体)。 因此我们可以将顶点数据分成两个槽:一个槽包含位置和纹理坐标,另一个槽包含其他顶点属性(例如,法线和切线矢量)。 现在我们可以轻松地仅在阴影传递所需的顶点数据中流动(位置和纹理坐标),从而为阴影传递节省数据带宽。 主渲染过程将使用两个顶点输入槽来获取所需的所有顶点数据。 为了提高性能,建议最小化用于小于或等于3的小数字的输入槽数。\\
\end{flushleft}
\begin{flushleft}
解:\\
要点:\\
\begin{itemize}
  \item 1. 调用 IASetVertexBuffers 指定输入槽用于提供数据
  \item 2. D3D12\_INPUT\_LAYOUT\_DESC的D3D12\_INPUT\_ELEMENT\_DESC 结构中的 inputSlot 用于指定 hlsl 中对应的 SemanticName 输入参数来自于哪个槽。
  \item 3. 1,2 提供了 Color 和 Pos 绑定在一起的依据。在顶点着色器阶段,能顺利将 Color 和 Pos 整合成 VertexIn。
\end{itemize}
需要修改的部分源码如下:\\
\end{flushleft}
\begin{lstlisting}
// chapter 6: BoxApp.cpp 修改
// 下面修改满足题目要求外,新增了一个三角形平面
// ...code...
struct VPosData
{
    XMFLOAT3 Pos;
};

struct VColorData
{
    XMFLOAT4 Color;
};

// ...code...

class BoxApp : public D3DApp
{
    // ...code...

    // 这里的 mTheta 和 mPhi 是初始时,摄像机对准的角度
    float mTheta = 1.5f*XM_PI; // 左右角度
    float mPhi = XM_PIDIV4;    // 上下角度
    // 摄像机到目标原点的距离,也相当于屏幕横坐标的大小;
    // 控制半径,能让立方体放大缩小,半径越大,立方体越小
    float mRadius = 10.0f;

    // ...code...
};

// ...code...

void BoxApp::Draw(const GameTimer& gt)
{
    // ...code...
    
    // 重要修改在这里

    // Position slot 0
    mCommandList->IASetVertexBuffers(0, 1, &mGeoPos->VertexBufferView());
    // Color: slot 1
    mCommandList->IASetVertexBuffers(1, 1, &mGeoColor->VertexBufferView());
    mCommandList->IASetIndexBuffer(&mGeoPos->IndexBufferView());
    mCommandList->IASetPrimitiveTopology(
                      D3D11_PRIMITIVE_TOPOLOGY_TRIANGLELIST);

    mCommandList->SetGraphicsRootDescriptorTable(0, 
                      mCbvHeap->GetGPUDescriptorHandleForHeapStart());

    SubmeshGeometry boxmesh = mGeoPos->DrawArgs["box"];
    mCommandList->DrawIndexedInstanced(boxmesh.IndexCount, 1,
                      boxmesh.StartIndexLocation, 
                      boxmesh.BaseVertexLocation, 0);

    SubmeshGeometry trianglemesh = mGeoPos->DrawArgs["triangle"];
    mCommandList->DrawIndexedInstanced(trianglemesh.IndexCount, 1, 
                      trianglemesh.StartIndexLocation, 
                      trianglemesh.BaseVertexLocation, 0);

    // ...code...
}

// ...code...

void BoxApp::BuildShadersAndInputLayout()
{
    HRESULT hr = S_OK;

    mvsByteCode = d3dUtil::CompileShader(L"Shaders\\color.hlsl", 
                      nullptr, "VS", "vs_5_0");
    mpsByteCode = d3dUtil::CompileShader(L"Shaders\\color.hlsl", 
                      nullptr, "PS", "ps_5_0");

    // 重要修改在这里: COLOR 的 InputSlot 为 1,AlignedByteOffset 为 0
    mInputLayout = {
        { "POSITION", 0, DXGI_FORMAT_R32G32B32_FLOAT, 0, 0, 
                      D3D12_INPUT_CLASSIFICATION_PER_VERTEX_DATA, 0 },
        { "COLOR",    0, DXGI_FORMAT_R32G32B32A32_FLOAT, 1, 0, 
                      D3D12_INPUT_CLASSIFICATION_PER_VERTEX_DATA, 0 }
    };
}

// 重要修改在这里:新增了三角形,position 数据和 color 数据分开
void BoxApp::BuildBoxGeometry()
{
    std::array<VPosData, 11> posVertices = {
        // 正立方体 position
        XMFLOAT3(-1.0f, -1.0f, -1.0f),  // 0
        XMFLOAT3(-1.0f, +1.0f, -1.0f),  // 1
        XMFLOAT3(+1.0f, +1.0f, -1.0f),  // 2
        XMFLOAT3(+1.0f, -1.0f, -1.0f),  // 3
        XMFLOAT3(-1.0f, -1.0f, +1.0f),  // 4
        XMFLOAT3(-1.0f, +1.0f, +1.0f),  // 5
        XMFLOAT3(+1.0f, +1.0f, +1.0f),  // 6
        XMFLOAT3(+1.0f, -1.0f, +1.0f),  // 7
        // 三角形 color
        XMFLOAT3(2.0f, 1.0f, -2.0f),    // 0
        XMFLOAT3(2.0f, +3.0f,-2.0f),    // 1
        XMFLOAT3(+3.0f, +3.0f, -2.0f)   // 2
    };

    std::array<VColorData, 11> colorVertices = {
        // 正立方体 color
        XMFLOAT4(Colors::White),  // 0
        XMFLOAT4(Colors::Black),  // 1
        XMFLOAT4(Colors::Red),    // 2
        XMFLOAT4(Colors::Green),  // 3
        XMFLOAT4(Colors::Blue),   // 4
        XMFLOAT4(Colors::Yellow), // 5
        XMFLOAT4(Colors::Cyan),   // 6
        XMFLOAT4(Colors::Magenta),// 7
        // 三角形 color
        XMFLOAT4(Colors::Magenta), // 0
        XMFLOAT4(Colors::Blue),    // 1
        XMFLOAT4(Colors::Yellow)   // 2
    };

    std::array<std::uint16_t, 39> indices = {
        // 正立方体面
        // front face
        0, 1, 2,
        0, 2, 3,

        // back face
        4, 6, 5,
        4, 7, 6,

        // left face
        4, 5, 1,
        4, 1, 0,

        // right face
        3, 2, 6,
        3, 6, 7,

        // top face
        1, 5, 6,
        1, 6, 2,

        // bottom face
        4, 0, 3,
        4, 3, 7, // indexCount:36

        /*
        因为 DrawIndexedInstanced 里定义了 BaseVertexLocation,
        trianglemesh.BaseVertexLocation = 8,
        所以索引又从 0 开始来描绘三角形。即这里的 0,1,2 代表 vertices[8],[9],[10]
        */
        // 三角形
        0, 1, 2  // indexCount: 3
    };

    const UINT vbPosByteSize = 
              (UINT)posVertices.size() * sizeof(VPosData);
    const UINT vbColorByteSize = 
              (UINT)colorVertices.size() * sizeof(VColorData);
    const UINT ibByteSize = 
              (UINT)indices.size() * sizeof(std::uint16_t);

    mGeoPos = std::make_unique<MeshGeometry>();
    mGeoPos->Name = "GeoPos";

    mGeoColor = std::make_unique<MeshGeometry>();
    mGeoColor->Name = "GeoColor";

    ThrowIfFailed(D3DCreateBlob(vbPosByteSize, 
                  &mGeoPos->VertexBufferCPU));
    CopyMemory(mGeoPos->VertexBufferCPU->GetBufferPointer(), 
               posVertices.data(), vbPosByteSize);

    ThrowIfFailed(D3DCreateBlob(vbColorByteSize, 
                      &mGeoColor->VertexBufferCPU));
    CopyMemory(mGeoColor->VertexBufferCPU->GetBufferPointer(), 
               colorVertices.data(), vbColorByteSize);

    ThrowIfFailed(D3DCreateBlob(ibByteSize, 
                     &mGeoPos->IndexBufferCPU));
    CopyMemory(mGeoPos->IndexBufferCPU->GetBufferPointer(), 
                   indices.data(), ibByteSize);

    // color 不用 indexBuffer

    mGeoPos->VertexBufferGPU = d3dUtil::CreateDefaultBuffer(
        md3dDevice.Get(),
        mCommandList.Get(), posVertices.data(), 
        vbPosByteSize, mGeoPos->VertexBufferUploader);

    mGeoPos->IndexBufferGPU = d3dUtil::CreateDefaultBuffer(
        md3dDevice.Get(),
        mCommandList.Get(), indices.data(),
        ibByteSize, mGeoPos->IndexBufferUploader);

    mGeoColor->VertexBufferGPU = d3dUtil::CreateDefaultBuffer(
        md3dDevice.Get(),
        mCommandList.Get(), colorVertices.data(),
        vbColorByteSize, mGeoColor->VertexBufferUploader);

    mGeoPos->VertexByteStride = sizeof(VPosData);
    mGeoPos->VertexBufferByteSize = vbPosByteSize;
    mGeoPos->IndexFormat = DXGI_FORMAT_R16_UINT;
    mGeoPos->IndexBufferByteSize = ibByteSize;

    mGeoColor->VertexByteStride = sizeof(VColorData);
    mGeoColor->VertexBufferByteSize = vbColorByteSize;
    mGeoColor->IndexFormat = DXGI_FORMAT_R16_UINT;
    mGeoColor->IndexBufferByteSize = ibByteSize;

    SubmeshGeometry boxmesh;
    boxmesh.IndexCount = (UINT)36;
    boxmesh.StartIndexLocation = 0;
    boxmesh.BaseVertexLocation = 0;

    SubmeshGeometry trianglemesh;
    trianglemesh.IndexCount = (UINT)3;
    trianglemesh.StartIndexLocation = 36;
    trianglemesh.BaseVertexLocation = 8;

    mGeoPos->DrawArgs["box"] = boxmesh;
    mGeoPos->DrawArgs["triangle"] = trianglemesh;

    mGeoColor->DrawArgs["box"] = boxmesh;
    mGeoColor->DrawArgs["triangle"] = trianglemesh;
}
// ...code...
\end{lstlisting}

%---------- 3 ----------------
\begin{flushleft}
3. 绘制以下图形:\\
\begin{itemize}
  \item 1. 一个点列表(point list),如图5.13a。
  \item 2. 一个线段条(line strip),如图5.13b。
  \item 3. 一个线段列表(line list),如图5.13c。
  \item 4. 一个三角条(triangle strip),如图5.13d。
  \item 5. 一个三角列表(triangle list),如图5.14a。
\end{itemize}
解: \\
\begin{itemize}
  \item 1. 修改 BoxApp::Draw 方法:\\
  \begin{lstlisting}
  // D3D11_PRIMITIVE_TOPOLOGY_TRIANGLELIST 改为 
  // D3D11_PRIMITIVE_TOPOLOGY_POINTLIST
  mCommandList->IASetPrimitiveTopology(
      D3D11_PRIMITIVE_TOPOLOGY_POINTLIST);
  \end{lstlisting}
  \item 2. 同 1 改为 D3D11\_PRIMITIVE\_TOPOLOGY\_LINESTRIP 即可
  \item 3. 同 1 改为 D3D11\_PRIMITIVE\_TOPOLOGY\_LINELIST 即可
  \item 4. 同 1 改为 D3D11\_PRIMITIVE\_TOPOLOGY\_TRIANGLESTRIP 即可
  \item 5. 同 1 改为 D3D11\_PRIMITIVE\_TOPOLOGY\_TRIANGLELIST 即可
\end{itemize}

\end{flushleft}

%---------- 4 ----------------
\begin{flushleft}
4. 构造金字塔的顶点和索引列表,如图\ref{fig:6-8}所示,并绘制它。 将基顶点设为绿色和顶上顶点设为红色。\\
解:修改 BoxApp.cpp 中金字塔的顶点和索引列表。代码如下
\end{flushleft}
\begin{lstlisting}
void BoxApp::BuildBoxGeometry()
{
    // DirectX 是左手坐标系,z轴负在前,正在后
    std::array<Vertex, 5> vertices = {
        // 金字塔
        Vertex({ XMFLOAT3(-1.0f, 0.0f, -1.0f), 
                 XMFLOAT4(Colors::Green) }),  // 0: (-1,0,-1)
        Vertex({ XMFLOAT3(+1.0f, 0.0f, -1.0f), 
                 XMFLOAT4(Colors::Green) }),  // 1: (1,0,-1)
        Vertex({ XMFLOAT3(+1.0f, 0.0f, +1.0f), 
                 XMFLOAT4(Colors::Green) }),  // 2: (1,0,1)
        Vertex({ XMFLOAT3(-1.0f, 0.0f, +1.0f), 
                 XMFLOAT4(Colors::Green) }),  // 3: (-1,0,1)
        Vertex({ XMFLOAT3(0.0f, +1.5f, 0.0f),  
                 XMFLOAT4(Colors::Red) }),    // 4: (0,1.5,0)
    };

    std::array<std::uint16_t, 18> indices = {
        // 底面
        0, 1, 2,
        0, 2, 3,

        // 侧面1
        0, 4, 1,

        // 侧面2
        1, 4, 2,

        // 侧面3
        2, 4, 3,

        // 侧面4
        3, 4, 0
    };

    const UINT vbByteSize = (UINT)vertices.size() * sizeof(Vertex);
    const UINT ibByteSize = (UINT)indices.size() * sizeof(std::uint16_t);

    mGeo = std::make_unique<MeshGeometry>();
    mGeo->Name = "Geo";

    ThrowIfFailed(D3DCreateBlob(vbByteSize, 
                      &mGeo->VertexBufferCPU));
    CopyMemory(mGeo->VertexBufferCPU->GetBufferPointer(), 
               vertices.data(), vbByteSize);

    ThrowIfFailed(D3DCreateBlob(ibByteSize, 
                      &mGeo->IndexBufferCPU));
    CopyMemory(mGeo->IndexBufferCPU->GetBufferPointer(), 
               indices.data(), ibByteSize);

    mGeo->VertexBufferGPU = d3dUtil::CreateDefaultBuffer(
                                md3dDevice.Get(),
                                mCommandList.Get(), 
                                vertices.data(), 
                                vbByteSize, 
                                mGeo->VertexBufferUploader);

    mGeo->IndexBufferGPU = d3dUtil::CreateDefaultBuffer(
                               md3dDevice.Get(),
                               mCommandList.Get(), 
                               indices.data(), 
                               ibByteSize, 
                               mGeo->IndexBufferUploader);

    mGeo->VertexByteStride = sizeof(Vertex);
    mGeo->VertexBufferByteSize = vbByteSize;
    mGeo->IndexFormat = DXGI_FORMAT_R16_UINT;
    mGeo->IndexBufferByteSize = ibByteSize;

    SubmeshGeometry pyramidmesh;
    pyramidmesh.IndexCount = (UINT)indices.size();
    pyramidmesh.StartIndexLocation = 0;
    pyramidmesh.BaseVertexLocation = 0;

    mGeo->DrawArgs["pyramid"] = pyramidmesh;
}
\end{lstlisting}

\begin{figure}[h]
    \includegraphics[width=\textwidth]{6-8}
    \centering
    \caption{金字塔三角形}
    \label{fig:6-8}
\end{figure}
%---------- 5 ----------------
\begin{flushleft}
5. 运行“Box” DEMO,并回想一下我们仅在顶点指定颜色。 解释如何为三角形上的每个像素获取像素颜色。\\
解:(猜测)在像素着色阶段时,使用了多采样技术(具体见 4.1.7节),取像素平均值。由于 Box 所使用的颜色简单,多采样使用 1 倍采样做处理即可。(d3dApp.h中m4xMsaaState默认为false)
\end{flushleft}

%---------- 6 ----------------
\begin{flushleft}
6. 修改 Box Demo, 在转换为世界空间之前将以下变换应用于顶点着色器中的每个顶点。
\end{flushleft}
\begin{lstlisting}
vin.PosL.xy += 0.5f*sin(vin.PosL.x)*sin(3.0f*gTime);
vin.PosL.z *= 0.6f + 0.4f*sin(2.0f*gTime);
\end{lstlisting}
\begin{flushleft}
您需要添加一个gTime常量缓冲区变量; 此变量对应于当前的 GameTimer::TotalTime() 值。 使用正弦函数周期性地扭曲顶点来将时间通过顶点动画来展现。\\
解:根据要求作如下修改:
\end{flushleft}
\begin{lstlisting}
//------ BoxApp.cpp 结构声明 ObjectConstants -------
struct ObjectConstants
{
    XMFLOAT4X4 WorldViewProj = MathHelper::Identity4x4();
    float gTime; // 新增一个 gTime 属性
};

//------- BoxApp.cpp BoxApp::Update 增加一行 gTime 赋值语句 -----
void BoxApp::Update(const GameTimer& gt)
{
    //...code...
    objConstants.gTime = mTimer.TotalTime();
    mObjectCB->CopyData(0, objConstants);
}

//------ BoxApp.cpp BoxApp::BuildConstantBuffers ------ 
// 将 UploadBuffer elementCount 参数改为 2
void BoxApp::BuildConstantBuffers()
{
    mObjectCB = std::make_unique<UploadBuffer<ObjectConstants>>(
                    md3dDevice.Get(), 2, true);

    //...code...
}

//------ color.hlsl 修改 cbPerObject 结构 ------
cbuffer cbPerObject : register(b0)
{
    float4x4 gWorldViewProj;
    float gTime;
};

//----- color.hlsl 修改 VS ------
VertexOut VS(VertexIn vin)
{
    vin.PosL.xy += 0.5f*sin(vin.PosL.x)*sin(3.0f*gTime);
    vin.PosL.z *= 0.6f + 0.4f*sin(2.0f*gTime);

    VertexOut vout;

    // Transform to homogeneous clip space.
    vout.PosH = mul(float4(vin.PosL, 1.0f), gWorldViewProj);
    
    // Just pass vertex color into the pixel shader.
    vout.Color = vin.Color;

    return vout;
}
\end{lstlisting}

%---------- 7 ----------------
\begin{flushleft}
7. 将box和金字塔的顶点(练习4)合并到一个大的顶点缓冲区中。 还将框和金字塔的索引合并到一个大索引缓冲区中(但不更新索引值)。 然后使用 ID3D12CommandList::DrawIndexedInstanced 的参数逐个绘制框和金字塔。 使用世界变换矩阵,使box和金字塔在世界空间中不相交。//
~\\
解:重点在常量缓冲区的更新,设置两个常量缓冲区,构建好各自的 D3D12\_CONSTANT\_BUFFER\_VIEW\_DESC,分别代表box 和 pyramid 的 worldViewProj 再做渲染
SetGraphicsRootDescriptorTable 绑定不同的常量缓冲区视图。\\
下面代码的修改的效果,切换视角时,图形间有相交的情况(待解决)。
\end{flushleft}
\begin{lstlisting}
//------ 修改 BoxApp.cpp BoxApp::Update 方法------
void BoxApp::Update(const GameTimer& gt)
{
    // Convert Spherical to Cartesian coordinates
    float x = mRadius * sinf(mPhi) * cosf(mTheta);
    float z = mRadius * sinf(mPhi) * sinf(mTheta);
    float y = mRadius * cosf(mPhi);

    XMVECTOR pos = XMVectorSet(x, y, z, 1.0f); // 摄像机位置
    XMVECTOR up = XMVectorSet(0.0f, 1.0f, 0.0f, 0.0f);

    XMMATRIX world = XMLoadFloat4x4(&mWorld);
    XMMATRIX proj = XMLoadFloat4x4(&mProj);
    
    // 正面摄像机摆放时,box 坐标为 (-2,0,0)
    XMVECTOR boxPos     = XMVectorSet(-2.0f, 0.0f, 0.0f, 0.0f); 
    // 正面摄像机摆放时,pyramid 坐标为 (+2,0,0)
    XMVECTOR pyramidPos = XMVectorSet(+2.0f, 0.0f, 0.0f, 0.0f); 

    // Build the view matrix.
    // Box 世界视图投影
    XMMATRIX view = XMMatrixLookAtLH(pos, boxPos, up);
    XMStoreFloat4x4(&mView, view);
    XMMATRIX worldViewProj = world * view * proj;

    // Update the constant buffer with 
    // the latest worldViewProj matrix.
    ObjectConstants objConstants;
    XMStoreFloat4x4(&objConstants.WorldViewProj, 
        XMMatrixTranspose(worldViewProj));
    mObjectCB->CopyData(0, objConstants);

    // pyramid 世界视图投影
    view = XMMatrixLookAtLH(pos, pyramidPos, up);
    XMStoreFloat4x4(&mView, view);

    worldViewProj = world * view * proj;
    XMStoreFloat4x4(&objConstants.WorldViewProj, 
        XMMatrixTranspose(worldViewProj));
    mObjectCB->CopyData(1, objConstants);
}
//------ 修改 BoxApp.cpp BoxApp::Draw 方法------
void BoxApp::Draw(const GameTimer& gt)
{
    //...code...

    UINT objCBByteSize = d3dUtil::CalcConstantBufferByteSize(
        sizeof(ObjectConstants));
    CD3DX12_GPU_DESCRIPTOR_HANDLE cbvHandle(
        mCbvHeap->GetGPUDescriptorHandleForHeapStart());

    mCommandList->SetGraphicsRootDescriptorTable(0, cbvHandle);
    SubmeshGeometry boxmesh = mGeo->DrawArgs["box"];
    mCommandList->DrawIndexedInstanced(
        boxmesh.IndexCount, 1,
        boxmesh.StartIndexLocation,
        boxmesh.BaseVertexLocation, 0);

    cbvHandle.Offset(1, mCbvSrvUavDescriptorSize);
    mCommandList->SetGraphicsRootDescriptorTable(0, cbvHandle);
    SubmeshGeometry pyramidmesh = mGeo->DrawArgs["pyramid"];
    mCommandList->DrawIndexedInstanced(
        pyramidmesh.IndexCount, 1,
        pyramidmesh.StartIndexLocation, 
        pyramidmesh.BaseVertexLocation, 0);

    //...code...
}
//------ 修改 BoxApp.cpp BoxApp::BuildDescriptorHeaps 方法------
void BoxApp::BuildDescriptorHeaps()
{
    D3D12_DESCRIPTOR_HEAP_DESC cbvHeapDesc;
    cbvHeapDesc.NumDescriptors = 2; // 1 改为 2
    //...code...
}
//------ 修改 BoxApp.cpp BoxApp::BuildConstantBuffers 方法------
void BoxApp::BuildConstantBuffers()
{
    mObjectCB = std::make_unique<UploadBuffer<ObjectConstants>>(
        md3dDevice.Get(), 2, true); // 1 改为 2

    UINT objCBByteSize = d3dUtil::CalcConstantBufferByteSize(sizeof(ObjectConstants));

    D3D12_GPU_VIRTUAL_ADDRESS cbAddress = mObjectCB->Resource()->GetGPUVirtualAddress();
    // Offset to the ith object constant buffer in the buffer.
    int boxCBufIndex = 0;
    cbAddress += boxCBufIndex * objCBByteSize;

    D3D12_CONSTANT_BUFFER_VIEW_DESC boxCbvDesc;
    boxCbvDesc.BufferLocation = cbAddress;
    boxCbvDesc.SizeInBytes = objCBByteSize;

    int pyramidCBufIndex = 1;
    cbAddress += pyramidCBufIndex * objCBByteSize;

    D3D12_CONSTANT_BUFFER_VIEW_DESC pyramidCbvDesc;
    pyramidCbvDesc.BufferLocation = cbAddress;
    pyramidCbvDesc.SizeInBytes = objCBByteSize;

    CD3DX12_CPU_DESCRIPTOR_HANDLE cpuHandle(mCbvHeap->GetCPUDescriptorHandleForHeapStart());
    md3dDevice->CreateConstantBufferView(&boxCbvDesc, cpuHandle);
    cpuHandle.Offset(1, mCbvSrvUavDescriptorSize);
    md3dDevice->CreateConstantBufferView(&pyramidCbvDesc, cpuHandle);
}
//------ 修改 BoxApp.cpp BoxApp::BuildBoxGeometry 方法------
void BoxApp::BuildBoxGeometry()
{
    // DirectX 是左手坐标系,z轴负在前,正在后
    std::array<Vertex, 13> vertices = {
        // 立方体
        Vertex({ XMFLOAT3(-1.0f, -1.0f, -1.0f), XMFLOAT4(Colors::White) }),  // 0: (-1,-1,-1)
        Vertex({ XMFLOAT3(-1.0f, +1.0f, -1.0f), XMFLOAT4(Colors::Black) }),  // 1: (-1, +1, -1)
        Vertex({ XMFLOAT3(+1.0f, +1.0f, -1.0f), XMFLOAT4(Colors::Red) }),    // 2: (+1, +1, -1)
        Vertex({ XMFLOAT3(+1.0f, -1.0f, -1.0f), XMFLOAT4(Colors::Green) }),  // 3: (+1, -1, -1)
        Vertex({ XMFLOAT3(-1.0f, -1.0f, +1.0f), XMFLOAT4(Colors::Blue) }),   // 4: (-1, -1, +1)
        Vertex({ XMFLOAT3(-1.0f, +1.0f, +1.0f), XMFLOAT4(Colors::Yellow) }), // 5: (-1, +1, +1)
        Vertex({ XMFLOAT3(+1.0f, +1.0f, +1.0f), XMFLOAT4(Colors::Cyan) }),   // 6: (+1, +1, +1)
        Vertex({ XMFLOAT3(+1.0f, -1.0f, +1.0f), XMFLOAT4(Colors::Magenta) }),// 7: (+1, -1, +1)

        // 金字塔
        Vertex({ XMFLOAT3(-1.0f, 0.0f, -1.0f), XMFLOAT4(Colors::Green) }),  // 0: (-1,0,-1)
        Vertex({ XMFLOAT3(+1.0f, 0.0f, -1.0f), XMFLOAT4(Colors::Green) }),  // 1: (1,0,-1)
        Vertex({ XMFLOAT3(+1.0f, 0.0f, +1.0f), XMFLOAT4(Colors::Green) }),  // 2: (1,0,1)
        Vertex({ XMFLOAT3(-1.0f, 0.0f, +1.0f), XMFLOAT4(Colors::Green) }),  // 3: (-1,0,1)
        Vertex({ XMFLOAT3(0.0f, +1.5f, 0.0f),  XMFLOAT4(Colors::Red) }),    // 4: (0,1.5,0)
    };

    std::array<std::uint16_t, 54> indices = {
        // 立方体
        // 前面
        0, 1, 2,
        0, 2, 3,

        // 背面
        4, 6, 5,
        4, 7, 6,

        // 左侧面
        4, 5, 1,
        4, 1, 0,

        // 右侧面
        3, 2, 6,
        3, 6, 7,

        // 上面
        1, 5, 6,
        1, 6, 2,

        // 下面
        4, 0, 3,
        4, 3, 7, // indexCount: 36

        // 金字塔
        // 底面
        0, 1, 2, 
        0, 2, 3,

        // 侧面1
        0, 4, 1,

        // 侧面2
        1, 4, 2,

        // 侧面3
        2, 4, 3,

        // 侧面4
        3, 4, 0 // indexCount: 18
    };

    const UINT boxIndexCount = 36;
    const UINT boxStartIndexLocation = 0;
    const UINT boxBaseVertexLocation = 0;

    const UINT pyramidIndexCount = 18;
    const UINT pyramidStartIndexLocation = 36;
    const UINT pyramidBaseVertexLocation = 8;

    const UINT vbByteSize = (UINT)vertices.size() * sizeof(Vertex);
    const UINT ibByteSize = (UINT)indices.size() * sizeof(std::uint16_t);

    mGeo = std::make_unique<MeshGeometry>();
    mGeo->Name = "Geo";

    ThrowIfFailed(D3DCreateBlob(vbByteSize, &mGeo->VertexBufferCPU));
    CopyMemory(mGeo->VertexBufferCPU->GetBufferPointer(), vertices.data(), vbByteSize);

    ThrowIfFailed(D3DCreateBlob(ibByteSize, &mGeo->IndexBufferCPU));
    CopyMemory(mGeo->IndexBufferCPU->GetBufferPointer(), indices.data(), ibByteSize);

    mGeo->VertexBufferGPU = d3dUtil::CreateDefaultBuffer(md3dDevice.Get(),
        mCommandList.Get(), vertices.data(), vbByteSize, mGeo->VertexBufferUploader);

    mGeo->IndexBufferGPU = d3dUtil::CreateDefaultBuffer(md3dDevice.Get(),
        mCommandList.Get(), indices.data(), ibByteSize, mGeo->IndexBufferUploader);

    mGeo->VertexByteStride = sizeof(Vertex);
    mGeo->VertexBufferByteSize = vbByteSize;
    mGeo->IndexFormat = DXGI_FORMAT_R16_UINT;
    mGeo->IndexBufferByteSize = ibByteSize;

    SubmeshGeometry boxmesh;
    boxmesh.IndexCount = boxIndexCount;
    boxmesh.StartIndexLocation = boxStartIndexLocation;
    boxmesh.BaseVertexLocation = boxBaseVertexLocation;

    SubmeshGeometry pyramidmesh;
    pyramidmesh.IndexCount = pyramidIndexCount;
    pyramidmesh.StartIndexLocation = pyramidStartIndexLocation;
    pyramidmesh.BaseVertexLocation = pyramidBaseVertexLocation;

    mGeo->DrawArgs["box"] = boxmesh;
    mGeo->DrawArgs["pyramid"] = pyramidmesh;
}
\end{lstlisting}

%---------- 8 ----------------
\begin{flushleft}
8. 通过在线框(wireframe)模式下渲染多维数据集来修改Box演示。\\
解:题目意思没理解错的话,修改 BoxApp::BuildPSO() 中的 RasterizerState 的值
FillMode 设为 D3D12\_FILL\_MODE\_WIREFRAME 即可。
\end{flushleft}

%---------- 9 ----------------
\begin{flushleft}
9. 修改Box演示, 禁用背面剔除(D3D12\_CULL\_NONE); 也尝试剔除正面而不是背面(D3D12\_CULL\_FRONT)。 以线框(wireframe)模式输出结果,以便您可以更轻松地查看差异。\\
解:修改 BoxApp::BuildPSO() 中的 RasterizerState 的值即可。例如:
\end{flushleft}
\begin{lstlisting}
CD3DX12_RASTERIZER_DESC rsDesc(D3D12_DEFAULT);
rsDesc.CullMode = D3D12_CULL_MODE_NONE;

psoDesc.RasterizerState = rsDesc;
\end{lstlisting}

%---------- 10 ----------------
\begin{flushleft}
10. 如果顶点内存很重要,那么从128位颜色值减少到32位颜色值可能是值得的。修改“Box”演示 在顶点结构中使用32位颜色值而不是128位颜色值。 您的顶点结构和相应的顶点输入描述将如下所示:\\
\end{flushleft}
\begin{lstlisting}
struct Vertex
{
    XMFLOAT3 Pos;
    XMCOLOR Color;
}

D3D12_INPUT_ELEMENT_DESC vertexDesc[] = {
    {“POSITION”, 0, DXGI_FORMAT_R32G32B32_FLOAT, 0, 0, 
                 D3D12_INPUT_PER_VERTEX_DATA, 0},
    {“COLOR”,    0, DXGI_FORMAT_B8G8R8A8_UNORM, 0, 12,
                 D3D12_INPUT_PER_VERTEX_DATA, 0}
};
\end{lstlisting}
\begin{flushleft}
我们使用 DXGI\_FORMAT\_B8G8R8A8\_UNORM 格式(8位红色,绿色,蓝色和alpha)。 此格式对应于常见的32位图形颜色格式ARGB,但 DXGI\_FORMAT 符号以小端表示法列出它们在内存中出现的字节。 在little-endian中,多字节(multi-byte)数据字(word)的字节从最低有效字节写入最高有效字节,这就是为什么 ARGB 在内存中出现为 BGRA,其中最小内存地址处的最低有效字节和最高有效字节为 最高的内存地址。\\
解:略。
\end{flushleft}

%---------- 11 ----------------
\begin{flushleft}
11. 思考下面 C++ 顶点结构:
\end{flushleft}
\begin{lstlisting}
struct Vertex
{
    XMFLOAT3 Pos;
    XMFLOAT4 Color;
};
\end{lstlisting}
\begin{itemize}
  \item 1. 输入布局描述顺序是否需要匹配顶点结构顺序? 也就是说,以下顶点声明是否适用于此顶点结构? 做一个实验来找出答案。 然后给出你为什么认为它有效或无效的推理。
  \begin{lstlisting}
  D3D11_INPUT_ELEMENT_DESC vertexDesc[] =
  {
      {“COLOR”,    0, DXGI_FORMAT_R32G32B32A32_FLOAT, 0, 12,
                   D3D11_INPUT_PER_VERTEX_DATA, 0},
      {“POSITION”, 0, DXGI_FORMAT_R32G32B32_FLOAT, 0, 0,
                   D3D11_INPUT_PER_VERTEX_DATA, 0}
  };
  \end{lstlisting}
  \item 2. 相应的顶点着色器结构顺序是否需要匹配 C++ 顶点结构顺序? 也就是说,以下顶点着色器结构是否与上述 C++ 顶点结构一起使用? 做一个实验来找出答案。 然后给出你为什么认为它有效或无效的推理。
  \begin{lstlisting}
  struct VertexIn
  {
      float4 Color : COLOR;
      float3 Pos   : POSITION;
  };
  \end{lstlisting}
\end{itemize}
\begin{flushleft}
解:1. 输入布局描述(inputLayoutDescriptor)不需要匹配定点结构顺序。D3D12\_INPUT\_ELEMENT\_DESC 的 SemanticName,SemanticIndex,AlignedByteOffset 已经给出了足够的信息去匹配对应的定点结构,与顺序无关也没有影响。\\
2.同样没有影响,但同样的 semanticName 的参数需要顺序一致,因为SemanticIndex是按顺序匹配的
\end{flushleft}

%---------- 12 ----------------
\begin{flushleft}
12. 将视口(viewport)设置为后缓冲区(back buffer)的左半部分。\\
解:
\end{flushleft}
\begin{lstlisting}
// 修改 BoxApp::OnResize() 即可
void BoxApp::OnResize()
{
    D3DApp::OnResize();
    mScreenViewport.Width = mClientWidth / 2;

    // The window resized, so update the aspect ratio and recompute the projection
    // matrix.
    XMMATRIX P = XMMatrixPerspectiveFovLH(0.25f*MathHelper::Pi, AspectRatio(), 1.0f, 1000.0f);
    XMStoreFloat4x4(&mProj, P);
}
\end{lstlisting}

%---------- 13 ----------------
\begin{flushleft}
13. 使用剪刀测试来剔除以后缓冲区为中心的矩形外的所有像素,宽度为mClientWidth/2,高度为 mClientHeight/2。 请记住,您还需要使用光栅化器状态组启用剪刀测试。\\
解:\\
CD3DX12\_RASTERIZER\_DESC 里并没有 ScissorEnable 属性,应该是书里写错了。
\end{flushleft}
\begin{lstlisting}
//----- 修改 BoxApp::OnResize 即可------
void BoxApp::OnResize()
{
    D3DApp::OnResize();
    mScissorRect = { 0, 0, mClientWidth / 2, mClientHeight / 2};

    // The window resized, so update the aspect ratio and recompute the projection
    // matrix.
    XMMATRIX P = XMMatrixPerspectiveFovLH(0.25f*MathHelper::Pi, AspectRatio(), 1.0f, 1000.0f);
    XMStoreFloat4x4(&mProj, P);
}
\end{lstlisting}

%---------- 14 ----------------
\begin{flushleft}
14. 像素着色器颜色色调。 使用常量缓冲区为颜色随时间变化。 使用smooth easing方法。 在顶点着色器和像素着色器中执行此操作。\\
解:参考第 6 题,将 gTime 放入常量缓冲区中,然后在 hlsl 使用 \href{https://docs.microsoft.com/en-us/windows/desktop/direct3dhlsl/dx-graphics-hlsl-lerp}{\textcolor{linkColor}{lerp}}方法(网上找了没有smooth easing方法, 应该是指lerp)做颜色计算
\end{flushleft}

%---------- 15 ----------------
\begin{flushleft}
15. 修改 Box DEMO 中的像素着色器为如下形式:
\end{flushleft}
\begin{lstlisting}
float4 PS(VertexOut pin) : SV_Target
{
    clip(pin.Color.r - 0.5f);
    return pin.Color;
}
\end{lstlisting}
\begin{flushleft}
运行程序并猜测内置 clip 方法的作用。\\
解: 现象,图被切成一块一块的。\\
文档见\href{https://docs.microsoft.com/en-us/windows/desktop/direct3dhlsl/dx-graphics-hlsl-clip}{\textcolor{linkColor}{clip}}\\
"如果x参数的每个分量表示与平面的距离,则使用剪辑HLSL内部函数来模拟剪裁平面。
此外,使用剪辑功能测试alpha行为"
\end{flushleft}

%---------- 16 ----------------
\begin{flushleft}
16.修改 Box 演示中的像素着色器,以在插值顶点颜色和通过常量缓冲区指定的 gPulseColor 之间平滑脉冲。 您还需要更新应用程序端的常量缓冲区。 HLSL代码中的常量缓冲区和像素着色器应如下所示:
\end{flushleft}
\begin{lstlisting}
cbuffer cbPerObject : register(b0)
{
    float4x4 gWorldViewProj;
    float4 gPulseColor;
    float gTime;
};

float4 PS(VertexOut pin) : SV_Target
{
    const float pi = 3.14159;

    // Oscillate a value in [0,1] over time using a sine functio.
    float s = 0.5f*sin(2*gTime - 0.25f(pi) + 0.5f;

    // Linearly interpolate between pin.Color and gPulseColor based on
    // parameter s.
    float4 c = lerp(pin.Color, gPulseColor, s);
    return c;
}
\end{lstlisting}
\begin{flushleft}
gTime 变量对应于 GameTimer::TotalTime() 的值。\\
解:略。
\end{flushleft}
