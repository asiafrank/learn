\documentclass[11pt,a4paper,oldfontcommands]{memoir}
\usepackage{ctex}
\usepackage[utf8]{inputenc}
\usepackage[T1]{fontenc}
\usepackage{microtype}
\usepackage[dvips]{graphicx}
\usepackage{xcolor}
\usepackage{times}

\usepackage[
breaklinks=true,colorlinks=true,
%linkcolor=blue,urlcolor=blue,citecolor=blue,% PDF VIEW
linkcolor=black,urlcolor=black,citecolor=black,% PRINT
bookmarks=true,bookmarksopenlevel=2]{hyperref}

\usepackage{geometry}
% PDF VIEW
% \geometry{total={210mm,297mm},
% left=25mm,right=25mm,%
% bindingoffset=0mm, top=25mm,bottom=25mm}
% PRINT
\geometry{total={210mm,297mm},
left=20mm,right=20mm,
bindingoffset=10mm, top=25mm,bottom=25mm}

\OnehalfSpacing
\chapterstyle{bianchi}
\setsecheadstyle{\Large\bfseries\sffamily\raggedright}
\setsubsecheadstyle{\large\bfseries\sffamily\raggedright}
\setsubsubsecheadstyle{\bfseries\sffamily\raggedright}

\pagestyle{plain}
\makepagestyle{plain}
\makeevenfoot{plain}{\thepage}{}{}
\makeoddfoot{plain}{}{}{\thepage}
\makeevenhead{plain}{}{}{}
\makeoddhead{plain}{}{}{}
\maxsecnumdepth{subsection} % chapters, sections, and subsections are numbered
\maxtocdepth{subsection} % chapters, sections, and subsections are in the Table of Contents

\begin{document}
\thispagestyle{empty}
{
\sffamily
\centering
{\LARGE
Introduction To 3D Game Programming with DirectX12 笔记
}

\vspace{3.5cm}
(草稿)
\clearpage
\tableofcontents*
\clearpage
\chapter{向量代数(Vector Algebra)}
\chapter{矩阵代数}
\chapter{矩阵变换}
\chapter{Direct3D 初始化}

\chapter{渲染管道(The Rendering Pipeline)}
\section{3D图像}
\section{模型表达}
\section{基本计算颜色}
\begin{flushleft}
计算机显示器在每个像素发射一种混合红、绿、蓝三种颜色的光。当这种混合光到达眼睛并且击中视网膜区域,视锥细胞收到刺激并将产生的神经冲动通过视觉神经传递到大脑。大脑解释这种信号为颜色。随着混合光的变化,使这些视锥细胞收到不同的刺激,从而使大脑中产生不同的颜色。
\end{flushleft}

\subsection{颜色计算}
\begin{itemize}
	\item 加法:$(0.0, 0.5, 0) + (0, 0.0, 0.25) = (0.0, 0.5, 0.25)$
	\item 减法:$(1, 1, 1) - (1, 1, 0) = (0, 0, 1)$
	\item 标量乘法:$0.5(1, 1, 1) = (0.5, 0.5, 0.5)$
	\item 显然,点乘和叉乘对于颜色向量来说没有意义。然而,颜色向量有一特殊的运算称为调制或分量乘法:$(c_{r},c_{g},c_{b}) \otimes (k_{r},k_{g},k_{b}) = (c_{r}k_{r},c_{g}k_{g},c_{b}k_{b})$。该运算主要用来作为照明方程。举个例子,假设我们有一束入射光线(r,g,b),它会照射一个反射50%红光,75%绿光和25%蓝光的表面,并吸收剩余的光线。 然后反射光线的颜色由下式给出:$(r,g,b) \otimes (0.5,0.75,0.25) = (0.5r,0.75g,0.25b)$
\end{itemize}

\end{document}

